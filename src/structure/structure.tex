% !TEX root=/home/tavant/these/manuscript/src/manuscript.tex




\chapter*{Proposition de plan}

Proposition de plan, le \today.
\linebreak

{\bf Concepts and preliminaries} 10 pages max
\begin{zzz}
  Instead of an introduction, this beforehand gives the background.
  \begin{itemize}
    \item Present the HET context (economical, strategic and political background)
    \item Overview of the Simulation techniques for plasmas
    \item few words on POSEIDON and Safran's X00 project
  \end{itemize}
\end{zzz}

{\bf I. Particle in Cell simulations} 20 pages
\begin{zzz}
  1.1 Elements of the 2D PIC-MCC simulations

  1.2 Simulating a 3D system with 2D plans : the {\bf $R-\theta$} and {\bf $Z-\theta$} cases presentations, hypotheses

  1.3 Modeling Dielectrics : Poisson equation and SEE

  1.4 {\it Fake} axial convection in {\bf $R-\theta$} simulations
\end{zzz}

{\bf II. Parametric study of the dielectric} 30 pages
\begin{zzz}
  This chapter takes the 1rst paper which uses Vivien's results.

  2.1 Fully metallic wall (no SEE, grounded).

  2.2 Impact of Dielectric layer without SEE

  2.3 Impact of SEE with grounded wall

  2.4 SEE and dielectric in the same time

  2.5 Discrepancy between $\mean{\Te}$, $\sigma_{PIC}$ and $\sigma_{theo} = \sigma_0 + (1 - \sigma_0) \frac{2 T_e}{\epsilon_0}$
\end{zzz}

{\bf III. Polytrotic sheath model} 23 pages
\begin{zzz}
  This chapter takes the 2nd paper about the modified sheath model

  3.1 EVDF in the 2D PIC simulations of HET.    3 pages

  3.2 1D simplified simulations, Parametric study and polytropic fits. 7 pages

  3.3 Monte-Carlo simulations.  3 pages

  3.4 fluid equations with polytropic closure. 10 pages
\end{zzz}

{\bf IV. Polytrotic sheath model with SEE} 20 pages
\begin{zzz}
  This chapter goes beyond the actual modified sheath model in order to add SEEs.
  This require some time to develop before writting !

  5.1 Kinetic effects of the SEE on the EVDFs  4 pages

  5.2 SEE effects on the Fluid model (using a 3 fluid model)  10 pages

  5.3 Validation against the Parametric 2D PIC simulation results. 4 pages
\end{zzz}

{\bf V. analyse of the instability } 30 pages
\begin{zzz}
  \begin{itemize}
\item Instability dispersion relation
\item Kinetic solver for Ion and ECDI using VDFs
\item Impact of the difference w/r Maxwellian
\item Wall Boundary condition, 3D dispertion relation vs 1D and 2D relations
\item Linear stage as ECDI and Saturation toward IAW
\end{itemize}
\end{zzz}


{\bf VI. another ? } 20 pages

Une derniere 6eme partie peut etre ajoutee, si besoin / temps/ volontee.

\linebreak
{\bf Conclusion } 5 pages
\begin{zzz}
The Conclusion
\end{zzz}

\vspace{2em}
{\bf Peut etre ajoute:}\\
Certains sujet peut etres abordees, mais necessite du temps.
\begin{itemize}
  \item Model non isoterme des gaines dans le code fluid 1D axial (type Barral)
  \item Le Z-tetha et le Fake R
  \item Le model Global de la simulation R-theta updated (cf these Vivien)
  \item Comparaison avec experiments
  \item Dielectric with RF
\end{itemize}
%
% \section{First part : $R-\theta$}
%
% \begin{itemize}
%   \item Reinjection : noise and impact
%   \item Dielectric
%   \item Gaine non-isotherme
%   \item Theorie RSO
%   \item Model global
% \end{itemize}
%
% \section{Second part : $Z-\theta$}
%
% \begin{itemize}
%   \item Fake R
%   \item ECDI theorie
%   \item Comparaison ML
% \end{itemize}

%
% Teste
%
% $\mean{A_{3}}$ \unit{5}{\meter}
%
%
%
% \ac{ABC}
%
% %
% \begin{equation}
%   a = 12
% \end{equation}
% \nomenclature{$a$}{The number of angels per unit area }%
%
%
% \cite{lucken2018}
