% !TEX root=/home/tavant/these/manuscript/src/manuscript.tex



\section{Insights for the PIC simulations}
\label{sec-insights}

As announced in \vref{sec-sheath_validation}, the sheath model of \vref{sec-sheath} uses two hypothesis\string:
\begin{itemize}
  \item Maxwellian electrons,
  \item Isothermal evolution of the electrons.
\end{itemize}

When collisions can be neglected, as it is usually assumed in the sheath, these two hypothesis are linked.
Indeed, the \ac{1D} Maxwellian distribution function expressed as the total energy is
\begin{equation} \label{eq-maxw_total}
  f(\ek, \phi) \propto \exp \lp \frac{\ek - \phi}{\Te}  \rp = \propto \exp \lp \frac{\ek}{\Te} \frac{-\phi}{\Te},
\end{equation}
where $\ek$ and $\Te$ are the electron kinetic energy and temperature expressed in Volt.
We can see in \cref{eq-maxw_total} that the spatial variation (due to the plasma potential $\phi$) only affect the amplitude of the distribution function, not its shape in the energy space.
Hence, the electron temperature is uniform, i.e. they are isotherm.
In addition, we find that $n_e \propto \exp (- \phi / \Te)$, which is the definition of Boltzmann electrons.

Hence, let see if this two aspects are respected in the \ac{2D} \ac{PIC}-\ac{MCC} simulation results

\subsection{Electron distribution function}

\begin{figure}[hbtp]
  \centering
  \includegraphics[width=\defaultwidth]{EEDF_2-eps-converted-to}
  \caption{Electron energy distribution function of the electrons ({\bf a}) in the bulk, in the three directions, and ({\bf b}) in the bulk and in the sheath}
  \label{fig-EEDF}
\end{figure}

Using the kinetic information of the PIC simulations, we present in \Cref{fig-EEDF} the mean electron energy probability functions (EEPF) in the case $\crover = 200\,\volt$.
\Cref{fig-EEDF}.{\bf a} shows the projections of the EEPF in the centre of the simulations along the three directions.
These projections are compared to the Maxwellian probability function of the same
kinetic temperature.
\Cref{fig-EEDF}.{\bf b} shows the total EEPF for both the bulk and the sheath populations.
 The sheath length is defined as the location where the ions reach the Bohm speed, which is about 0.4mm.
 
 We see in \cref{fig-EEDF}.{\bf a} that the electron distribution function is not Maxwellian.
 In particular the high energy tails are depleted.
 However, we can see that the electrons are rather isotropic, compared to \ac{1D} simulation \citep{sydorenko2006}.
 In order to evaluate the effect of the nonMaxwellian EEPF, we numerically integrate the EEPF from the PIC data using \vref{eq-ratedifinition_evdf}.
The results (not shown) do not differ significantly from the Maxwellian values of \vref{eq-seemaxw}.
Hence, we can conclude that even if the Maxwellian hypothesis is not respected in the
PIC simulations, it is not enough to explain the differences observed in \vref{fig-seeparamesMaxw}.


\Cref{fig-EEDF}.{\bf b} presents the EEPF for the bulk population as well as for the sheath population.
 We can see that the sheath population is colder than the population at the centre, which could explain the difference of \vref{fig-seeparamesMaxw}. 
 This effect is assessed in the next section.



 