% !TEX root=/home/tavant/these/manuscript/src/manuscript.tex



\section{Simplified 1D for the PIC simulations}
\label{sec-1DPIC}

We have seen in \cref{sec-insights} that the secondary electron emission is not responsible for the temperature gradient.
Hence, we will neglect it is this section.
The simulations were at low pressure, in which case  the electrons are non-local \cite{bernstein1954, godyak1993}.
In low pressure bounded discharges, it is well-known that the EEDF is not Maxwellian in both capacitively coupled plasmas and inductively coupled plasmas \cite{mouchtouris2016, godyak2002, meige2006a, dominguez-vazquez2018}, in agreement whith \vref{fig-EEDF}.

The impact of non-Maxwellian EEDF on the electron flux at the wall has been studied by Kaganovich et al. \citep{kaganovich2000,kaganovich2007}.
They showed that the electron kinetics at low pressure can significantly reduce the electron flux to the wall, in agreement with kinetic simulations.
The main parameter determining the electron flux was found to be the electron scattering frequency.
However, to the knowledge of the authors no model describes the sheath with non-Maxwellian EEDF, that could be used in fluid equations.

The evolution of the electron temperature and the non-locality of the electron in bounded plasmas as been studied in \citet{meige2006a}.
Hence, we will use in this chapter similar physical conditions, in order to compare with there conclusions.
We use a 1D PIC simulation of an argon plasma confined between two walls separated by a length $L=10$cm.
The background pressure is varied between 0.05 and 10 mTorr.

The same particle source model as in \ac{2D}\ac{PIC} of the \ac{HET} is used.
In order to compensate the particle losses at the wall, we inject with a spatially uniform probability an electron-ion couple for every ion lost at the wall.
This corresponds to the following ionization rate:
\begin{equation}
  S_{iz} = \frac{1}{L} 2 \Gamma_e
\end{equation}
with $\Gamma_e$ the electron flux to the wall.
A second model will be used latter, with a self-consistent heating and ionization.

Monte Carlo collisions (MCC) are still used, but we do not model the particle generation of the ionization process, but only the scattering and momentum transfer.
As previously, Coulomb collisions are not included in the study as we are at low plasma density (at the steady state the electron density is around $n_e = 10^{15}$m$^{-3}$).

To satisfy generally accepted accuracy conditions for the cell size and time step \citep{turner2013}, a time step of $3.7\cdot10^{-11}$\,s is used with a cell length of $1.7\cdot10^{-5}$\,m.
This allows us to resolve properly the plasma frequency $\frac{2 \pi}{\omega_{pe}} = 3.5\cdot10^{-9} $\,s and the Debye length $\lambda_{De} = 3.10^{-4}$\,m.
Around $300$ particles per cell are used for the simulations, and statistical convergence has been verified for both the cell length and the number of particles per cell.


\begin{table}
  \centering
  \begin{tabular}{lll}  \toprule
    Parameter & Value & Unit  \\ \midrule
    Pressure $P$ & $0.05,0.1,0.5, 2, 10$ & mTorr\\
    Initial density & $1 .10^{15}$ & m$^{-3}$\\
    $\Te_{, inj}$& 5 & V\\
    Domain length $L$ & 10 & cm\\
    gas & Argon & -\\
    \bottomrule
  \end{tabular}
  \caption{Simulation parameters for the 1D PIC simulations.}
  \label{tab_1DPICParams}
\end{table}










