% !TEX root=/home/tavant/these/manuscript/src/manuscript.tex

\section{Conclusion }
\label{sec-ch3conclusion}

Using the kinetic informations from the \ac{PIC} simulations, we have seen that the electrons are not Maxwellian, in contrast with the hypothesis of the usual models for plasma wall interactions.
The electron distribution function is affected by two phenomena\string:
\begin{itemize}
  \item the absorption of high energy electrons at the wall
  \item the electron-neutral scattering
\end{itemize}
\vspace{1em}
The absorption depletes rapidly the high energy tail of the EEPF for energies higher than the local plasma potential relative to the wall.
However, the low energy population is not affected by the wall.

The collisions affect the electrons more slowly, by replenishing the high energy tail by scattering.
Indeed, in the directions parallel to the wall, the high energy tail is not depleted.
However, for large energies ($\ek > 10\,\volt$), the electron-neutral scattering angle is small \citep{vahedi1995}, hence the time scale over which the collisions impact the EEPF is much longer than the typical time between two collisions.

The electron trajectory in the discharge chamber is hence mostly collisionless.
We have successfully confirmed this by confronting the EEDF measurements to the 1D stationary Vlasov equation.
Following the work of \citet{zhang2016} on the collisionless evolution of non-Maxwellian electron through a potential drop, we have found that a polytropic closure for the electron describes very accurately the electron temperature evolution\string:
\begin{equation*} \label{eq-polyp2}
  \Te n_e^{1-\gamma} = cst, \text{ with $\gamma$ the polytropic index}
\end{equation*}

The polytropic state law for the electrons, when used in fluid model, allows to obtain the same densities and plasma potential as in the \ac{PIC} simulation.
This paves the way for a modified sheath model to compare to the \ac{2D} \ac{PIC} simulation of the \ac{HET} of \cref{ch-2}.

We have also seen in \cref{sec-realistic_1D} that the polytropic state law also stands when a self-consistent heating mechanism is used, even if the agreement is not as good as in the other case.

In \cref{eq-polyp2}, the value of the polytropic index $\gamma$ depends on the shape of the \ac{EVDF}.
We showed in \cref{sec-MCM} that a Monte Carlo computation can be used in order to obtain the \ac{EVDF} for a given plasma potential profile and neutral pressure.
As the Monte Carlo approach does not need the Poisson equation to be solved, it produces the \ac{EVDF} much faster than a \ac{PIC} simulation.
Hence, we could couple the Monte Carlo calculation with a fluid model to accurately take into account the real shape of the \ac{EVDF} in the closure of terms in fluid models.

