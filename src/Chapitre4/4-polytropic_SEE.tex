% !TEX root=/home/tavant/these/manuscript/src/manuscript.tex




\chapter{Secondary electron emission with anisothermal sheath}
\label{ch-5}

\begin{Chabstract}
  
We modify the polytropic sheath model with secondary electron emission
\end{Chabstract}

{\bf IV. Polytrotic sheath model with SEE} 20 pages
\begin{zzz}
  This chapter goes beyond the actual modified sheath model in order to add SEEs.
  This require some time to develop before writting !

  5.1 Kinetic effects of the SEE on the EVDFs  4 pages

  5.2 SEE effects on the Fluid model  5 pages

  5.3 Validation against the Parametric 2D PIC simulation results. 2 pages
  
  5.4 1D fluid model with modified wall model. 4 pages
\end{zzz}

\minitoc

What is complicated here is the definition of the electron temperature, that include both primary and secondary electrons.
Should we
\begin{itemize}
  \item Use a 2 fluid( so 1 electron) with a fluid model that includes the SEE ?
  \item Use a 3 fluid model, that would  include folly absorbed primary electrons (simply polytropic) and emitted electrons, with strictly positive velocity.
\end{itemize}

In the First case, the electron temperature is simple to define.
But the closure equation may need to be changed.

the second case is more simpler, mathematicaly.

References:
Il y a déjà eu beaucoup de travaux, mais aucun model fluides !

\citet{meezan2002} : Bolzmann solver, 2-Te EVDF, SEE compared with Maxwellian

\citet{smirnov2004} : MCC code, 2-Te EVDF. Mais fake EVDI effect via collisions

\citet{sydorenko2006a} : Beam in EVDF because of 1D PIC-MCC 

\citet{raitses2006} : "strong anisotropy : facteur 4",  compare Mesures avec fluids codes: says disagriment. Talks about 

\citet{ahedo2002} : sheath-presheath model with SEE with bolzmann electron, Sagdeev potential, SCL regime avec saturations 

\citet{ahedo2003} : 1D axial model avec wall interactions (SEE et SCL), plum with section increase (A(z)) 

\citet{ahedo2005} : Fluid model with SEE, with partial trapping and partial beams

\citet{raitses2005} : SCL regime and Te saturation (where he says that the SCL may not be responsible ??)

\citet{barral2003a} : 1D model, anisotrop electrons ({\bf must read to know how}), shows saturation at SCL with large anisotropy

{\bf \Large A lire}

\citet{sydorenko2007} :  non maxwellian EVDF

\citet{raitses2005a} : electron-wall interaction

\citet{jolivet2000} : SEE effect on EVDF 