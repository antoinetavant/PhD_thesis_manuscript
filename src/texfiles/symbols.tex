% !TEX root=/home/tavant/these/manuscript/src/manuscript.tex

\usepackage{xspace}
\xspaceaddexceptions{]\}}


%Uncomment next line if AMS fonts required
%\usepackage{iopams}
\newcommand{\prl}{\parallel}
\newcommand{\estar}{\epsilon^*}
\newcommand{\Te}{\mathrm{T_e}}
\newcommand{\Teb}{\mathrm{T_{e, 0}}}
\newcommand{\Tew}{\mathrm{T_{e, wall}}}
\newcommand{\Tepar}{\mathrm{T_e}_{\prl}}
\newcommand{\Ti}{\mathrm{T_i}}
\newcommand{\Ts}{\mathrm{T_s}}
\newcommand{\Tsee}{\mathrm{T_{SEE}}}
\newcommand{\sig}{\sigma\xspace}
\newcommand{\sigm}{\sigma_{\rm max}\xspace}
\newcommand{\sigo}{\sigma_0\xspace}

\newcommand{\norm}[1]{\lvert#1\rvert}
\renewcommand{\vec}[1]{{\bf #1}}
\newcommand{\eV}{{\mathrm{\,V}}}
\newcommand{\LPPicname}{{\it LPPic}}
\newcommand{ \red}{\color{red}}
\newcommand{\ek}{\epsilon}

\usepackage{upgreek}
\newcommand{\mus}{\mathrm{\upmu  s}}

\renewcommand{\div}{\nabla \cdot}

\newcommand{\lb}{\left[}
\newcommand{\rb}{\right]}

\newcommand{\lp}{\left(}
\renewcommand{\rp}{\right)}

\newcommand{\dphi}{\Delta \phi}
\newcommand{\dphisheath}{\Delta \phi_{\rm sheath}}
\newcommand{\mobe}{\mu_{\mathrm{e}}}
\newcommand{\mobeffsat}{\mu_{\mathrm{eff}}^{sat}}
\newcommand{\mobeff}{\mu_{\mathrm{eff}}}
\newcommand{\mobcla}{\mu_{\mathrm{classical}}}
\newcommand{\mobpic}{\mu_{\mathrm{PIC}}}
\newcommand{\M}[1]{{\bf M#1}}

\newcommand{\Ee}{{\mathcal{E}_e}}
\newcommand{\Es}{{\mathcal{E}_s}}

\newcommand\Npc{N_{\rm pc}}
\newcommand\Isp{\ensuremath {\rm I_{\rm sp}} \xspace}

% 10^number: use as \sn{Value}{Power} for value x 10^Power
\newcommand{\sn}[2]{\ensuremath{{{#1}{\times}10^{#2}}}}


%mean notation. May need an average command as well
%\newcommand{\mean}[1]{{\overline{#1}}}

\newcommand\LPPic{{\it LPPic} }

\newcommand\dt{\ensuremath{\Delta t} }


\newcommand{\vect}[1]{{\mathbf #1}}
\newcommand{\epsr}{\epsilon_{R}}
\newcommand{\V}{\Omega_{i,j}}
\newcommand{\C}{C_{i,j}}

\DeclareMathOperator{\grad}{\vect{\nabla}}
\newcommand{\deriv}[2]{\frac{\partial #1}{\partial #2}}

\newcommand{\dx}{\Delta x}
\newcommand{\dy}{\Delta y}

\newcommand{\N}{\ensuremath{\mathcal{N}}}
\newcommand\stdconv{\ensuremath{\sigma_{\rm Reinj}}}
\newcommand{\aziE}{ {\ensuremath{E_{\theta}}}}
\newcommand\FFT{\ensuremath{ \mathcal{FFT}}}
\newcommand\FT{\ensuremath{ \mathcal{FT}}}

%========================
% Nomenclature

%% This code creates the groups
% -----------------------------------------

\renewcommand\nomgroup[1]{%
  \item[\bfseries
  \ifstrequal{#1}{P}{Physics Constants}{%
  \ifstrequal{#1}{N}{Numerical Implementation}{%
  \ifstrequal{#1}{Q}{Quantities}{}}}%
]}

% This will add the units
%----------------------------------------------
\newcommand{\nomunit}[1]{%
\renewcommand{\nomentryend}{\hspace*{\fill}#1}}
%----------------------------------------------

\newcommand\PPS{PPS\textregistered }

\newcommand\proba{\ensuremath{\sigma }}
\newcommand\probamax{\ensuremath{\proba_{\rm max} }}

\newcommand\rate{\ensuremath{ \bar{\proba}}\xspace}
\newcommand\ratemaxw{\ensuremath{ \bar{\proba}_{\rm Maxw}}\xspace}
\newcommand\ratepic{\ensuremath{ \bar{\proba}_{\rm PIC}}\xspace}
\newcommand\ratecr{\ensuremath{ \bar{\proba}_{\rm cr}}\xspace}


\newcommand\mob{\ensuremath{\mu_e}}
\newcommand\mobunit{\ensuremath{ \text{m}^2\text{(Vs)}^{-1}}  }
\newcommand\oce{\ensuremath{\omega_{ce}}}
\newcommand\ope{\ensuremath{\omega_{pe}}}
\newcommand\opi{\ensuremath{\omega_{pi}}}
\newcommand\lde{\ensuremath{\lambda_{De}}}
\newcommand\crover{\ensuremath{\ek^{*}}}

\newcommand\stdE{\ensuremath{\sigma_{E_{\theta}}}}

\newcommand\dne{\ensuremath{\delta n_e}}
\newcommand\dEt{\ensuremath{\delta E_{\theta}}}
\newcommand\Rei{\ensuremath{R_{ei}}}
\newcommand\viout{\ensuremath{  v_{i, \rm out} }}

\let\arobase\at
\renewcommand{\at}[1]{ \ensuremath{ \bigg\vert_{#1} }}

\newcommand\dphiscl{\ensuremath{\dphi_{\rm SCL}}}


\newcommand\kms{\ensuremath{ \kilo\meter\per\second }}

\newcommand\ztheta{\ensuremath{ ( \mathbf Z - \theta) } }

%nomenclature text 
\renewcommand{\nompreamble}{In this work we use the SI system of units, except for a few units specific to the plasma community that are describe below. The notations in general follow common usage. The vectors are noted in bold, as $\vect{v}$, $\vect{E}$, etc.
Complex quantities are not distinguished from real ones.

The next list describes several symbols that will be later used within the body of the document.}

\nomenclature[P]{Torr}{Equivalent to one millimeter of mercury, \nomunit{$1$\,Torr = 133.32 \pascal}}
\nomenclature[P]{Gauss}{Unit of magnetic flux density in cgs (centimeter-gram-second units) \nomunit{$1$\,G = $\sn{1}{-4}\,\tesla $}}
\nomenclature[P]{eV}{Unit of energy, corresponds to the kinetic energy of one electron accelerated by an electric potential difference of one volt. It is usually used in plasma physics as a unit of temperature via the Boltzmann constant. \nomunit{$1$\,eV $\simeq \sn{1.602}{-19}\,\joule$  }}

\nomenclature[P]{\ensuremath{ K_B}}{Boltzmann constant \nomunit{$\sn{1.380649}{-23}\,\joule\per\kelvin$}}
\nomenclature[P]{\ensuremath{ e}}{Elementary charge \nomunit{$\sn{1.602176634}{-19}$\,C}}
