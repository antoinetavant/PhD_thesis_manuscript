% !TEX root=/home/tavant/these/manuscript/src/manuscript.tex

%%%%%%%%%%%%%%%%%%%%%%%%%%%%%%%%%%%%%%%%%%%%%%%%%%%%%%%%%%%%%%%%%%%%%%%%%%%%%%%%%%%%%%%%%%%%%%%%%%%%%%%%%%%%%%%%%%%%%%%%%%%%%%%%%%%%%%%%%%%%%%%%%%%%%%%%%%%%%%%%%%%%%%%
%%%%%%%%%%%%%%%%%%%%%%%%%%%%%%%%%%%%%%%%%%%%%%%%%%%%%%%%%%%%%%%%%%%%%%%%%%%%%%%%%%%%%%%%%%%%%%%%%%%%%%%%%%%%%%%%%%%%%%%%%%%%%%%%%%%%%%%%%%%%%%%%%%%%%%%%%%%%%%%%%%%%%%%
%%% Modèle pour la 1ère de couverture des thèses préparées à l'Université Paris-Saclay, basé sur le modèle produit par Guillaume BRIGOT / Template for back cover of thesis made at Université Paris-Saclay, based on the template made by Guillaume BRIGOT
%%% Mis à jour par Aurélien ARNOUX (École polytechnique)/ Updated by Aurélien ARNOUX (École polytechnique)
%%% Les instructions concernant chaque donnée à remplir sont données en bloc de commentaire / Rules to fill this file are given in comment blocks
%%% ATTENTION Ces informations doivent tenir sur une seule page une fois compilées / WARNING These informations must contain in no more than one page once compiled
%%%%%%%%%%%%%%%%%%%%%%%%%%%%%%%%%%%%%%%%%%%%%%%%%%%%%%%%%%%%%%%%%%%%%%%%%%%%%%%%%%%%%%%%%%%%%%%%%%%%%%%%%%%%%%%%%%%%%%%%%%%%%%%%%%%%%%%%%%%%%%%%%%%%%%%%%%%%%%%%%%%%%%%
%%% Version du 19 juillet 2018 (Merci à Hadrien VROYLANDT (Univ. Paris-Sud) pour ses suggestions et corrections)
%%%%%%%%%%%%%%%%%%%%%%%%%%%%%%%%%%%%%%%%%%%%%%%%%%%%%%%%%%%%%%%%%%%%%%%%%%%%%%%%%%%%%%%%%%%%%%%%%%%%%%%%%%%%%%%%%%%%%%%%%%%%%%%%%%%%%%%%%%%%%%%%%%%%%%%%%%%%%%%%%%%%%%%

% \renewcommand{\familydefault}{\sfdefault}
\geometry{
	left=16mm,
	top=30mm,
	right=16mm,
	bottom=30mm
}
\definecolor{bordeau}{rgb}{0.3515625,0,0.234375}

\setlength{\columnseprule}{0pt}
\setlength\columnsep{10pt}


\label{form}
%%%%%%%%%%%%%%%%%%%%%%%%%%%%%%%%%%%%%%%%%%%%%%%%%%%%%%%%%%%%%%%%%%%%%%%%%%%%%%%%%%%%%%%%%%%%%%%%%%%%%%%%%%%%%%%%%%%%%%%%%%%%%%%%%%%%%%%%%%%%%%%%%%%%%%%%%%%%%%%%%%%%%%%
%%%%%%%%%%%%%%%%%%%%%%%%%%%%%%%%%%%%%%%%%%%%%%%%%%%%%%%%%%%%%%%%%%%%%%%%%%%%%%%%%%%%%%%%%%%%%%%%%%%%%%%%%%%%%%%%%%%%%%%%%%%%%%%%%%%%%%%%%%%%%%%%%%%%%%%%%%%%%%%%%%%%%%%
%%% Formulaire / Form
%%% Remplacer les paramètres des \newcommand par les informations demandées / Replace \newcommand parameters by asked informations
%%%%%%%%%%%%%%%%%%%%%%%%%%%%%%%%%%%%%%%%%%%%%%%%%%%%%%%%%%%%%%%%%%%%%%%%%%%%%%%%%%%%%%%%%%%%%%%%%%%%%%%%%%%%%%%%%%%%%%%%%%%%%%%%%%%%%%%%%%%%%%%%%%%%%%%%%%%%%%%%%%%%%%%
%%%%%%%%%%%%%%%%%%%%%%%%%%%%%%%%%%%%%%%%%%%%%%%%%%%%%%%%%%%%%%%%%%%%%%%%%%%%%%%%%%%%%%%%%%%%%%%%%%%%%%%%%%%%%%%%%%%%%%%%%%%%%%%%%%%%%%%%%%%%%%%%%%%%%%%%%%%%%%%%%%%%%%%


\newcommand{\PhDTitle}{Plasma-wall interaction and electron transport in Hall Effect Thrusters} 	%% Titre de la thèse / Thesis title
\newcommand{\PhDname}{Antoine Tavant} 															%% Civilité, nom et prénom /  Civility, first name and name
\newcommand{\NNT}{20XXSACLXXXX} 															%% Numéro National de Thèse (donnée par la bibliothèque à la suite du 1er dépôt)/ National Thesis Number (given by the Library after the first deposit)

\newcommand{\ecodoctitle}{Ondes et Matière} 													%% Nom de l'ED. Voir site de l'Université Paris-Saclay / Full name of Doctoral School. See Université Paris-Saclay website
\newcommand{\ecodocacro}{EDOM}																%% Sigle de l'ED. Voir site de l'Université Paris-Saclay / Acronym of the Doctoral School. See Université Paris-Saclay website
\newcommand{\ecodocnum}{572} 																%% Numéro de l'école doctorale / Doctoral School number
\newcommand{\PhDspeciality}{Physique des Plasmas} 										%% Spécialité de doctorat / Speciality
\newcommand{\PhDworkingplace}{l'École polytechnique} 										%% Établissement de préparation / PhD working place \string: l'Université Paris-Sud, l'Université de Versailles-Saint-Quentin-en-Yvelines, l'Université d'Evry-Val-d'Essonne, l'Institut des sciences et industries du vivant et de l'environnement (AgroParisTech), CentraleSupélec,l'Ecole normale supérieure de Cachan, l'Ecole Polytechnique, l'Ecole nationale supérieure de techniques avancées, l'Ecole nationale de la statistique et de l’administration économique, HEC Paris, l'Institut d'optique théorique et appliquée, Télécom ParisTech, Télécom SudParis
\newcommand{\defenseplace}{Palaiseau} 											%% Ville de soutenance / Place of defense
\newcommand{\defensedate}{18 décembre 2019} 															%% Date de soutenance / Date of defense

%%% Établissement / Institution
%%% Si la thèse a été produite dans le cadre d'une co-tutelle, commenter la partie "Pas de co-tutelle" et décommenter la partie "Co-tutelle" / If the thesis has been prepared in guardianship, comment the part "Pas de co-tutelle" and uncomment the part "Co-tutelle"

	%%%%%%%%%%%%%%%%%%%%%%%%%
	%%% Pas de co-tutelle %%%
	%%%%%%%%%%%%%%%%%%%%%%%%%

\newcommand{\logoEtt}{blank}																%% NE PAS MODIFIER / DO NOT MODIFY
\newcommand{\vpostt}{0.1} 																	%% NE PAS MODIFIER / DO NOT MODIFY
\newcommand{\hpostt}{6}																		%% NE PAS MODIFIER / DO NOT MODIFY
\newcommand{\logoEt}{X} 																	%% Logo de l'établissement de soutenance. Indiquer le sigle / Institution logo. Indicate the acronym \string: AGRO, CENTSUP, ENS, ENSAE, ENSTA, HEC, IOGS, TPT, TSP, UEVE, UPSUD, UVSQ, X
\newcommand{\vpos}{0.1}																		%% À modifier au besoin pour aligner le logo verticalement / If needed, modify to align logo vertilcally
\newcommand{\hpos}{11}																		%% À modifier au besoin pour aligner le logo horizontalement / If needed, modify to align logo horizontaly

		%%%%%%%%%%%%%%%%%%
		%%% Co-tutelle %%%
		%%%%%%%%%%%%%%%%%%
%
% \newcommand{\logoEt}{etab} 																%% Logo de l'université partenaire. Placer le fichier .png dans le répertoire '/couvertures/etab' et indiquer le nom du fichier sans l'extension / Logo of partner university. Place the .png file in the directory '/couvertures/etab' and point the file name without the extension
% \newcommand{\vpos}{0.1}																	%% À modifier au besoin pour aligner les logos verticalement / If needed, modify to align logos vertilcally
% \newcommand{\hpos}{11}																		%% À modifier au besoin pour aligner les logos horizontalement / If needed, modify to align logos horizontaly
% \newcommand{\logoEtt}{etab2}  																%% Logo de l'établissement de soutenance. Le nom du fichier correspond au sigle de l'établissement /  Institution logo. Filename correspond to institution acronym \string: AGRO, CENTSUP, ENS, ENSAE, ENSTA, HEC, IOGS, TPT, TSP, UEVE, UPSUD, UVSQ, X
% \newcommand{\vpostt}{0.1} 																	%% À modifier au besoin pour aligner les logos verticalement / If needed, modify to align logos vertilcally
% \newcommand{\hpostt}{6}																	%% À modifier au besoin pour aligner les logos horizontalement / If needed, modify to align logos horizontaly


%%% JURY

% Lors du premier dépôt de la thèse le nom du président n’est pas connu, le choix du président se fait par les membres du Jury juste avant la soutenance. La précision est apportée sur la couverture lors du second dépôt / Choice of the jury's president is made during the defense. Thus, it must be specified only for the second file deposition in ADUM.
% Tous les membres du juty listés doivent avoir été présents lors de la soutenance / All the jury members listed here must have been present during the defense.

%%% Membre n°1 (Président) / Member n°1 (President)
\newcommand{\jurynameA}{Prof. Pere Roca}
\newcommand{\juryadressA}{Dir. de recherche, LPICM / CNRS}
\newcommand{\juryroleA}{Examinateur}

%%% Membre n°2 (Rapporteur) / Member n°2 (President)
\newcommand{\jurynameB}{Prof. Achim von Keudell}
\newcommand{\juryadressB}{Professeur, Ruhr-Universitäte Bochum}
\newcommand{\juryroleB}{Rapporteur}

%%% Membre n°3 (Rapporteur) / Member n°3 (President)
\newcommand{\jurynameC}{Dr. Gwenael Fubiani}
\newcommand{\juryadressC}{Ch. de recherche, LAPLACE / CNRS}
\newcommand{\juryroleC}{Rapporteur}

%%% Membre n°4 (Examinateur) / Member n°4 (President)
\newcommand{\jurynameD}{Dr. Sedina Tsikata}
\newcommand{\juryadressD}{Ch. de recherche, ICARE / CNRS}
\newcommand{\juryroleD}{Examinateur}

%%% Membre n°5 (Directeur de thèse) / Member n°5 (Thesis supervisor)
\newcommand{\jurynameE}{Dr. Anne Bourdon}
\newcommand{\juryadressE}{Dir. de recherche, LPP / CNRS}
\newcommand{\juryroleE}{Directeur de thèse}

%%% Membre n°6 (Co-directeur de thèse) / Member n°6 (Thesis co-supervisor)
\newcommand{\jurynameF}{Dr. Pascal Chabert}
\newcommand{\juryadressF}{Dir. de recherche, LPP / CNRS}
\newcommand{\juryroleF}{Co-directeur de thèse}

%%% Membre n°7 (Invité) / Member n°7 (Guest)
\newcommand{\jurynameG}{Dr. Stephan Zurbach}
\newcommand{\juryadressG}{Ing. expert senior, Safran Aircraft Engines}
\newcommand{\juryroleG}{Invité}

%%% Membre n°8 (Invité) / Member n°8 (Guest)
\newcommand{\jurynameH}{Prénom Nom}
\newcommand{\juryadressH}{Statut, Établissement (Unité de recherche)}
\newcommand{\juryroleH}{Invité}

%% Il est possible d'ajouter des membres supplémentaires selon le même modèle / More jury members can be added according to the same model

\label{layout}
%%%%%%%%%%%%%%%%%%%%%%%%%%%%%%%%%%%%%%%%%%%%%%%%%%%%%%%%%%%%%%%%%%%%%%%%%%%%%%%%%%%%%%%%%%%%%%%%%%%%%%%%%%%%%%%%%%%%%%%%%%%%%%%%%%%%%%%%%%%%%%%%%%%%%%%%%%%%%%%%%%%%%%%
%%%%%%%%%%%%%%%%%%%%%%%%%%%%%%%%%%%%%%%%%%%%%%%%%%%%%%%%%%%%%%%%%%%%%%%%%%%%%%%%%%%%%%%%%%%%%%%%%%%%%%%%%%%%%%%%%%%%%%%%%%%%%%%%%%%%%%%%%%%%%%%%%%%%%%%%%%%%%%%%%%%%%%%
%%% Mise en page / Page layout
%%% NE RIEN MODIFIER EXCEPTÉ LA PARTIE CONCERNANT LE JURY (voir \label{jury}) SI BESOIN / DO NOT MODIFY EXCEPT SECTION CONCERNING JURY (see \label{jury}) IF NEEDED
%%%%%%%%%%%%%%%%%%%%%%%%%%%%%%%%%%%%%%%%%%%%%%%%%%%%%%%%%%%%%%%%%%%%%%%%%%%%%%%%%%%%%%%%%%%%%%%%%%%%%%%%%%%%%%%%%%%%%%%%%%%%%%%%%%%%%%%%%%%%%%%%%%%%%%%%%%%%%%%%%%%%%%%
%%%%%%%%%%%%%%%%%%%%%%%%%%%%%%%%%%%%%%%%%%%%%%%%%%%%%%%%%%%%%%%%%%%%%%%%%%%%%%%%%%%%%%%%%%%%%%%%%%%%%%%%%%%%%%%%%%%%%%%%%%%%%%%%%%%%%%%%%%%%%%%%%%%%%%%%%%%%%%%%%%%%%%%

% Méta-données du PDF / PDF meta-datas
\hypersetup{
	pdfauthor={\PhDname},
	pdfsubject={Ph.D. thesis manuscrit},
	pdftitle={\PhDTitle},
}




\newcommand{\logoEd}{EDOM}																		%% Logo de l'école doctorale. Indiquer le sigle / Doctoral school logo. Indicate the acronym \string: 2MIB; AAIF; ABIES; BIOSIGNE; CBMS; EDMH; EDOM; EDPIF; EDSP; EOBE; INTERFACES; ITFA; PHENIICS; SDSV; SDV; SHS; SMEMAG; SSMMH; STIC
\newcommand{\PhDTitleFR}{Intéraction plasma-paroi et tranport des électrons dans les moteurs à effet Hall.}													%% Titre de la thèse en français / Thesis title in french
\newcommand{\keywordsFR}{Propulsion électrique, simulation PIC, modèle de gain non isothermal, instabilité de dérive cyclotronique}														%% Mots clés en français, séprarés par des , / Keywords in french, separated by ,
\newcommand{\abstractFR}{

Les moteurs électriques pour satellites, qui accélèrent un plasma, sont primordiaux pour le succès des missions spatiales (GPS, météo, communication, etc.).
Le moteur à effet Hall fait parti des technologies les plus performantes et utilisées.
Cependant, sa conception et optimisation est long et couteuse, car ma compris.
En particulier, le transport des électrons et l'interaction plasma-paroi. 
Afin d'étudier ces deux phénomènes, nous utilisons une simulation cinétique bidimensionnelle. 
 
Grace aux résultats de simulation, nous avons mis en évidence que les électrons sont non-locaux, car ils sont absorbé plus vite aux parois qu'ils ne sont thermalisé par les collisions.
En conséquence, nous avons développé un model de gaine avec une loi d'état polytropique pour les électrons, qui décrit plus précisément l'interaction plasma-surface.
Ce modèle peut être utilisé en présence, ou non, d'émission électronique secondaire.
Lorsque l'émission secondaire est présent, le model de gain présente jusqu'à trois solutions, qui explique les oscillations de gaines observées dans les simulations. 

L'instabilité azimutale observée, responsable du transport des électrons, est comparée aux relations de dispersions de l'instabilité acoustique ionique et l'instabilité cyclotronique de dérive électronique. 
Nous montrons que la phase de croissance linéaire est bien comprise, mais que l'état stationnaire dépend de l'interaction onde-particule et de phénomènes non-linéaires qui ne sont pas pris en compte dans les relations de dispersion. 

}															%% Résumé en français / abstract in french

\newcommand{\PhDTitleEN}{\PhDTitle}													%% Titre de la thèse en anglais / Thesis title in english
\newcommand{\keywordsEN}{Electric propulsion, PIC simulation, non-isothermal sheath model, cyclotron drift instability}														%% Mots clés en anglais, séprarés par des , / Keywords in english, separated by ,
\newcommand{\abstractEN}{

Electric propulsion systems that accelerate plasma are important for the success of spatial missions (GPS, weather forecast, communication, etc.).
The Hall effect thruster is one of the most used and efficient technology.
However, its conception and optimization is slow and costly, because poorly understood, in particular the electron transport and the plasma-wall interaction.
In order to study both phenomena, we use a bi-dimensional kinetic simulation. 

We showed with the PIC simulation results that electrons are non-local, as they are absorbed more quickly at the wall compared to the collision frequency.
Consequently, we derived a non-isothermal sheath model using a polytropic state law for the electrons that describes more accurately the plasma-wall interaction.
The model can be used with and without secondary electron emission.
With electron emission, the sheath model can present up to three solutions, explaining the oscillations observed in the simulations. 

The azimuthal instability observed, responsible for the electron transport, is compared to the dispersion relation of the ion acoustic wave and the electron cyclotron drift instability.
We show that, while the first linear stage of the instability is well understood, the saturated quasi-steady-state is affected by particle-wave interactions and non-linear mechanisms that are not included in the dispersion relation.
}															%% Résumé en anglais / abstract in english
