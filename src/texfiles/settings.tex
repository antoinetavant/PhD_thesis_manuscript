% !TEX root=/home/tavant/these/manuscript/src/manuscript.tex

\usepackage[english,french]{babel}
\shorthandoff{:}
% \usepackage{parskip}  %% Add spaces between paragraphs
% -------------
% font spec
%-----------
\usepackage[T1]{fontenc}

% \usepackage{ebgaramond}

\usepackage{titlesec}
\usepackage{titling}
% Specify different font for section headings
\newcommand*{\headingfont}{\fontfamily{phv}\selectfont}

\titleformat{\chapter}[display]{\Huge\headingfont}{\chaptertitlename\ \thechapter}{20pt}{\Huge}
\titleformat*{\section}{\LARGE\headingfont}
\titleformat*{\subsection}{\Large\headingfont}
\titleformat*{\subsubsection}{\large\headingfont}

%use \showfont to print the current font used
\makeatletter
\newcommand{\showfont}{encoding\string: \f@encoding{},
  family\string: \f@family{},
  series\string: \f@series{},
  shape\string: \f@shape{},
  size\string: \f@size{}
}
\newcommand{\iffont}[3]{\ifthenelse{\equal{\f@family}{#1}}{#2}{#3}}
\makeatother



\usepackage{lmodern}% http\string://ctan.org/pkg/lm for allowing arbitrary font size
\newcommand{\hsc}[1]{{\footnotesize\MakeUppercase{#1}}}   % Fake Small Caps

% \usepackage[latin1]{inputenc}
\usepackage[utf8]{inputenc}  % For french special characters 
\usepackage{eso-pic}	%eso-pic pour mettre des images avant les chapitres
\usepackage{float, caption}
\usepackage{amsmath,amssymb,mathrsfs,amsfonts}
\usepackage[amssymb]{SIunits}   %SI units, with dots for multiplications and medium space befor the unit

\usepackage[thmmarks,amsmath]{ntheorem}
\usepackage{array, multirow, tabularx}
\usepackage{mathrsfs}                   %pour faire des cursives dans les formules

\usepackage{empheq}  % better highlights in equations
\usepackage{cases}  % for cases in equations

\usepackage{listings}		        %Pour ecrire du code
\usepackage[final]{pdfpages}		%Pour insérer des feuilles pdf (page de garde+abstracts)
\usepackage{minitoc} 			%Pour faire des mini tables des matières en début de chapitre

\usepackage{xcolor} %pour avoir les couleurs

\usepackage{geometry}


\usepackage[nocfg]{nomencl}
\renewcommand{\nomname}{Nomenclature and List of Symbols}
\makenomenclature

\usepackage{textcomp} %pour les apostrophes
\usepackage[nottoc,numbib,notlof,notlot]{tocbibind} %pour afficher l'entrée biblio dans la table des matières
%\usepackage[firstpage]{draftwatermark} %pour marquer draft sur la première page
%\SetWatermarkScale{6} %taille de la watermark
%\SetWatermarkLightness{0.9} %transparence de la watermark


% GRAPHICS
\usepackage{graphicx}
\usepackage{subfig}	%pour mettre des subplots


\usepackage{etoolbox}
\patchcmd{\chapter}{plain}{empty}{}{}
% Remove warning from the bibliography URL too long
% Variant A
\apptocmd{\sloppy}{\hbadness 4492\relax}{}{}
% Variant B
% \apptocmd{\thebibliography}{\raggedright}{}{}

\usepackage{smartdiagram}


\definecolor{lightgray}{gray}{0.95}
\hyphenation{Fortran}		        %Pour eviter l'hyphenation de "Fortran"

\lstset{language=[90]Fortran,	%Pour bien ecrire le code en Fortran
  basicstyle=\fontsize{11}{13}\selectfont\ttfamily,
  keywordstyle=\color{gray},
  commentstyle=\color{blue},
  morecomment=[l]{!\ },  %Comment only with space after !
  captionpos=b, % sets the caption-position to bottom
  frame=single, % adds a frame around the code
  %numbers=left, % where to put the line-numbers
  %numbersep=5pt, % how far the line-numbers are from the code
  %stepnumber=2, % the step between two line-numbers
  rulecolor=\color{black}, % if not set, the frame-color may be changed
  backgroundcolor=\color{lightgray}, % choose the background color
  breaklines=true %break lines if too long
}

\usepackage[activate={true,nocompatibility},final,tracking=true,kerning=true,spacing=true,factor=1100,stretch=10,shrink=10]{microtype}
% activate={true,nocompatibility} - activate protrusion and expansion
% final - enable microtype; use "draft" to disable
% tracking=true, kerning=true, spacing=true - activate these techniques
% factor=1100 - add 10% to the protrusion amount (default is 1000)
% stretch=10, shrink=10 - reduce stretchability/shrinkability (default is 20/20)


% For better intext references. By example \cref{fig\string:X} produces "Fig. n° X" instead of just "X"
\usepackage{varioref} % add the page when the object is far away





% For better citations.
% \usepackage[firstinits=true]{biblatex}
\usepackage[numbers,
            ]{natbib}
%%-------------------> 
% \usepackage{citebackref} % optionnal
%%------------------->
\usepackage{bibentry} % Includ the bibliography in the text.
\usepackage{filecontents}
\nobibliography*
%~~~~~~~~~~~~~~~~~~~~~~~~~~~~~~
%  Overlay bar for means
\makeatletter
\newsavebox\myboxA
\newsavebox\myboxB
\newlength\mylenA
\usepackage[super]{nth}

\newcommand*\mean[2][0.75]{%
    \sbox{\myboxA}{$\m@th#2$}%
    \setbox\myboxB\null% Phantom box
    \ht\myboxB=\ht\myboxA%
    \dp\myboxB=\dp\myboxA%
    \wd\myboxB=#1\wd\myboxA% Scale phantom
    \sbox\myboxB{$\m@th\overline{\copy\myboxB}$}%  Overlined phantom
    \setlength\mylenA{\the\wd\myboxA}%   calc width diff
    \addtolength\mylenA{-\the\wd\myboxB}%
    \ifdim\wd\myboxB<\wd\myboxA%
       \rlap{\hskip 0.5\mylenA\usebox\myboxB}{\usebox\myboxA}%
    \else
        \hskip -0.5\mylenA\rlap{\usebox\myboxA}{\hskip 0.5\mylenA\usebox\myboxB}%
    \fi}
\makeatother


% !TEX root=/home/tavant/these/manuscript/src/manuscript.tex



\usepackage{xargs}                      % Use more than one optional parameter in a new commands
% \usepackage[pdftex,dvipsnames]{xcolor}  % Coloured text etc.
\usepackage[colorinlistoftodos,prependcaption,textsize=tiny]{todonotes}

\newcommandx{\unsure}[2][1=]{\todo[noline,linecolor=red,backgroundcolor=red!25,bordercolor=red,#1]{#2}}
\newcommandx{\change}[2][1=]{\todo[noline,linecolor=blue,backgroundcolor=blue!25,bordercolor=blue,#1]{#2}}
\newcommandx{\info}[2][1=]{\todo[noline,linecolor=OliveGreen,backgroundcolor=OliveGreen!25,bordercolor=OliveGreen,#1]{#2}}
\newcommandx{\improvement}[2][1=]{\todo[inline,linecolor=red,backgroundcolor=red!25,bordercolor=red,#1]{#2}}
\newcommandx{\thiswillnotshow}[2][1=]{\todo[noline,disable,#1]{#2}}
\newcommandx{\inlinenote}[2][1=]{\todo[inline,size=\small,#1]{#2}}



\usepackage{acronym}  %to handle acronym


% Define the abstract Env.
\usepackage{fancyhdr}
% \pagestyle{fancy}
% \pagestyle{plain}
% \pagestyle{empty}
% \fancyhf{}
% \fancyhead[LE,RO]{Overleaf}
% \fancyhead[RE]{\chaptername}
% \fancyhead[LE]{\thepage}

\newenvironment{abstract}%
    {\cleardoublepage\thispagestyle{empty}\null\vfill\begin{center}%
    \bfseries\abstractname\end{center}}%
    {\vfill\null}


% ignore bibliorapgy if empty
\let\myBib\thebibliography

\renewcommand\thebibliography[1]{\ifx\relax#1\relax\else\myBib{#1}\fi}

%for linebreak
\renewcommand\linebreak{\vspace{1em}}



\newenvironment{zzz}
               {\list{}{\rightmargin0pt \leftmargin4em }%
                \item\relax}
               {\endlist}

\newcommand\subfigurewidth{2.5in}
\newcommand\defaultwidth{3.5in}

\usepackage[]{algorithm2e}
 \usepackage{mathtools}


 %===================
 % Subfigures
\usepackage[percent]{overpic}
%Floats
\usepackage{placeins}  % allows \FloatBarrier

\newcommand\subfiguretics[3]{
\begin{overpic}[width=\subfigurewidth,grid,tics=10]{#1}
 \put (#3) {\bf #2}
\end{overpic}
}
\newcommand\subfigure[3]{
\begin{overpic}[width=\subfigurewidth]{#1}
 \put (#3) {\bf #2}
\end{overpic}
}


%% table
\usepackage{booktabs}
\newcommand{\ra}[1]{\renewcommand{\arraystretch}{#1}}


%% Fix acronym and cleveref

\makeatletter
% \newcommand*{\org@overidelabel}{}
% \let\org@overridelabel\@verridelabel
% \@ifpackagelater{acronym}{2015/03/21}{% v1.41
%   \renewcommand*{\@verridelabel}[1]{%
%     \@bsphack
%     \protected@write\@auxout{}{\string\AC@undonewlabel{#1@cref}}%
%     \org@overridelabel{#1}%
%     \@esphack
%   }%
% }{% older versions
%   \renewcommand*{\@verridelabel}[1]{%
%     \@bsphack
%     \protected@write\@auxout{}{\string\undonewlabel{#1@cref}}%
%     \org@overridelabel{#1}%
%     \@esphack
%   }%
% }

% 
% \def\@testdef #1#2#3{%
%   \def\reserved@a{#3}\expandafter \ifx \csname #1@#2\endcsname
%  \reserved@a  \else
% \typeout{^^Jlabel #2 changed\string:^^J%
% \meaning\reserved@a^^J%
% \expandafter\meaning\csname #1@#2\endcsname^^J}%
% \@tempswatrue \fi}
\makeatother


%  ==================
% Introduction
\usepackage{epigraph}

\newcommand\quotechapt[2]{
\begin{flushright}
  \begin{minipage}{8cm}
  \emph{ #2}
\rule{0.4\textwidth}{1pt}

#1
\end{minipage}
\end{flushright}
}

%\newenvironment{Chabstract}{\leftskip1in\itshape { }{ } }

\newenvironment{Chabstract} { \begin{quote}
\small   }
{\end{quote}}


\newcommand\headerchaptername[1]{
\renewcommand\leftmark{  \expandafter\MakeUppercase{ \chaptername\ \thechapter.\ #1}}
}

\newcommand\pageCompt[2]{ \ref{#1} has [\the\numexpr\getpagerefnumber{#2}-\getpagerefnumber{#1}\relax] pages in it. }

\ifpdf
  \usepackage[pagebackref=false,  % add the pages where the citation is used
              hyperindex=true,
              colorlinks=true]{hyperref}
  \hypersetup{pdfstartview={FitH}, bookmarksnumbered={true}}

\else
  \usepackage[hypertex=true,hyperindex=true,colorlinks=false]{hyperref}
\fi

\usepackage{bookmark}
\usepackage[absolute,overlay]{textpos}
\usepackage{graphicx}
\usepackage{array}
\usepackage{caption}
\usepackage{multicol}
\usepackage{xcolor}


\usepackage[capitalise]{cleveref}  %hance, uyse \vref for the page reference, else \cref only
\hypersetup{
    colorlinks,
    allcolors={blue!50!black},
    linkcolor={red!50!black},
    anchorcolor={red!50!black}
    citecolor={blue!50!black},
    urlcolor={blue!80!black},
    breaklinks
}