% !TEX root=/home/tavant/these/manuscript/src/manuscript.tex

\usepackage[english,french]{babel}


% -------------
% font spec
%-----------
\usepackage[T1]{fontenc}
% \usepackage{ebgaramond}

\usepackage{titlesec}
\usepackage{titling}
% Specify different font for section headings
\newcommand*{\headingfont}{\fontfamily{phv}\selectfont}

\titleformat*{\section}{\LARGE\headingfont}
\titleformat*{\subsection}{\Large\headingfont}
\titleformat*{\subsubsection}{\large\headingfont}

%use \showfont to print the current font used
\makeatletter
\newcommand{\showfont}{encoding: \f@encoding{},
  family: \f@family{},
  series: \f@series{},
  shape: \f@shape{},
  size: \f@size{}
}
\newcommand{\iffont}[3]{\ifthenelse{\equal{\f@family}{#1}}{#2}{#3}}
\makeatother



\usepackage{lmodern}% http://ctan.org/pkg/lm for allowing arbitrary font size
\newcommand{\hsc}[1]{{\footnotesize\MakeUppercase{#1}}}   % Fake Small Caps

\usepackage[latin1]{inputenc}
\usepackage{eso-pic}	%eso-pic pour mettre des images avant les chapitres
\usepackage{float, caption}
\usepackage{amsmath,amssymb,mathrsfs,amsfonts}
\usepackage[amssymb]{SIunits}

\usepackage[thmmarks,amsmath]{ntheorem}
\usepackage{array, multirow, tabularx}
\usepackage{mathrsfs}                   %pour faire des cursives dans les formules

\usepackage{SIunits}		        %SI units
\usepackage{listings}		        %Pour ecrire du code
\usepackage[final]{pdfpages}		%Pour insérer des feuilles pdf (page de garde+abstracts)
\usepackage{minitoc} 			%Pour faire des mini tables des matières en début de chapitre

\usepackage{xcolor} %pour avoir les couleurs



\usepackage[nocfg]{nomencl}
\renewcommand{\nomname}{List of Symbols}
\makenomenclature

\usepackage{textcomp} %pour les apostrophes
\usepackage[nottoc,numbib,notlof,notlot]{tocbibind} %pour afficher l'entrée biblio dans la table des matières
%\usepackage[firstpage]{draftwatermark} %pour marquer draft sur la première page
%\SetWatermarkScale{6} %taille de la watermark
%\SetWatermarkLightness{0.9} %transparence de la watermark


% GRAPHICS
\usepackage{graphicx}
\usepackage{subfig}	%pour mettre des subplots
\graphicspath{{images/}{Chapitre1/}{Chapitre1/figure/}{schemas/}{plots/}{images/}{logos/}}  % Location of the graphics files (set up for graphics to be in PDF format)


\usepackage{etoolbox}
\patchcmd{\chapter}{plain}{empty}{}{}


\definecolor{lightgray}{gray}{0.95}
\hyphenation{Fortran}		        %Pour eviter l'hyphenation de "Fortran"

\lstset{language=[90]Fortran,	%Pour bien ecrire le code en Fortran
  basicstyle=\fontsize{11}{13}\selectfont\ttfamily,
  keywordstyle=\color{gray},
  commentstyle=\color{blue},
  morecomment=[l]{!\ },  %Comment only with space after !
  captionpos=b, % sets the caption-position to bottom
  frame=single, % adds a frame around the code
  %numbers=left, % where to put the line-numbers
  %numbersep=5pt, % how far the line-numbers are from the code
  %stepnumber=2, % the step between two line-numbers
  rulecolor=\color{black}, % if not set, the frame-color may be changed
  backgroundcolor=\color{lightgray}, % choose the background color
  breaklines=true %break lines if too long
}

% TODO: Fix Greek mu

\usepackage[activate={true,nocompatibility},final,tracking=true,kerning=true,spacing=true,factor=1100,stretch=10,shrink=10]{microtype}
% activate={true,nocompatibility} - activate protrusion and expansion
% final - enable microtype; use "draft" to disable
% tracking=true, kerning=true, spacing=true - activate these techniques
% factor=1100 - add 10% to the protrusion amount (default is 1000)
% stretch=10, shrink=10 - reduce stretchability/shrinkability (default is 20/20)


% For better intext references. By example \cref{fig:X} produces "Fig. n° X" instead of just "X"
\usepackage{varioref} % add the page when the object is far away

\ifpdf
  \usepackage[pagebackref=true,hyperindex=true,colorlinks=true]{hyperref}
  \hypersetup{pdfstartview={FitH}, bookmarksnumbered={true}}

\else
  \usepackage[hypertex=true,hyperindex=true,colorlinks=false]{hyperref}
\fi
\hypersetup{
    colorlinks,
    linkcolor={red!50!black},
    citecolor={blue!50!black},
    urlcolor={blue!80!black}
}

\usepackage[capitalise]{cleveref}  %hance, uyse \vref for the page reference, else \cref only


% For better citations.
\usepackage[numbers]{natbib}
%\usepackage[style=phys]{biblatex}

%~~~~~~~~~~~~~~~~~~~~~~~~~~~~~~
%  Overlay bar for means
\makeatletter
\newsavebox\myboxA
\newsavebox\myboxB
\newlength\mylenA
\usepackage[super]{nth}

\newcommand*\mean[2][0.75]{%
    \sbox{\myboxA}{$\m@th#2$}%
    \setbox\myboxB\null% Phantom box
    \ht\myboxB=\ht\myboxA%
    \dp\myboxB=\dp\myboxA%
    \wd\myboxB=#1\wd\myboxA% Scale phantom
    \sbox\myboxB{$\m@th\overline{\copy\myboxB}$}%  Overlined phantom
    \setlength\mylenA{\the\wd\myboxA}%   calc width diff
    \addtolength\mylenA{-\the\wd\myboxB}%
    \ifdim\wd\myboxB<\wd\myboxA%
       \rlap{\hskip 0.5\mylenA\usebox\myboxB}{\usebox\myboxA}%
    \else
        \hskip -0.5\mylenA\rlap{\usebox\myboxA}{\hskip 0.5\mylenA\usebox\myboxB}%
    \fi}
\makeatother


% !TEX root=/home/tavant/these/manuscript/src/manuscript.tex



\usepackage{xargs}                      % Use more than one optional parameter in a new commands
% \usepackage[pdftex,dvipsnames]{xcolor}  % Coloured text etc.
\usepackage[colorinlistoftodos,prependcaption,textsize=tiny]{todonotes}

\newcommandx{\unsure}[2][1=]{\todo[noline,linecolor=red,backgroundcolor=red!25,bordercolor=red,#1]{#2}}
\newcommandx{\change}[2][1=]{\todo[noline,linecolor=blue,backgroundcolor=blue!25,bordercolor=blue,#1]{#2}}
\newcommandx{\info}[2][1=]{\todo[noline,linecolor=OliveGreen,backgroundcolor=OliveGreen!25,bordercolor=OliveGreen,#1]{#2}}
\newcommandx{\improvement}[2][1=]{\todo[inline,linecolor=red,backgroundcolor=red!25,bordercolor=red,#1]{#2}}
\newcommandx{\thiswillnotshow}[2][1=]{\todo[noline,disable,#1]{#2}}
\newcommandx{\inlinenote}[2][1=]{\todo[inline,size=\small,#1]{#2}}



\usepackage{acronym}  %to handle acronym


% Define the abstract Env.
\usepackage{fancyhdr}
\pagestyle{empty}

\newenvironment{abstract}%
    {\cleardoublepage\thispagestyle{empty}\null\vfill\begin{center}%
    \bfseries\abstractname\end{center}}%
    {\vfill\null}


% ignore bibliorapgy if empty
\let\myBib\thebibliography

\renewcommand\thebibliography[1]{\ifx\relax#1\relax\else\myBib{#1}\fi}

%for linebreak
\renewcommand\linebreak{\vspace{1em}}



\newenvironment{zzz}
               {\list{}{\rightmargin0pt \leftmargin4em }%
                \item\relax}
               {\endlist}

\newcommand\defaultwidth{4in}

\usepackage[]{algorithm2e}
 \usepackage{mathtools}
