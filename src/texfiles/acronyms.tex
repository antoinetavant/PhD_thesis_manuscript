% !TEX root=/home/tavant/these/manuscript/src/manuscript.tex

% More infor at http\string://ctan.mines-albi.fr/macros/latex/contrib/acronym/acronym.pdf

\chapter*{Acronyms}

\begin{acronym}
  \acro{HET}{Hall effect Thruster}


  \acro{PIC}{Particle in Cell}
  \acro{MCC}{Monte Carlo Collision}
  \acro{DK}{Direct Kinetic}
  \acro{LPP}{Laboratoire de Physique des plasmas \acroextra{\\ A laboratory from Ecole polytechnique, Palaiseau, France .}}
  \acro{Xe}{Xenon}
  \acro{Kr}{Krypton}

  \acro{3V}{3 dimension for the velocity  }
  \acro{2D}{2 dimensions  }
  \acro{3D}{3 dimensions  }
  \acro{1D}{1 dimension  }
  \acro{BNSiO2}{Boron Nitride-Silicon Dioxide\acroextra{. A ceramic composed of a mix of Boron Nitride and Silicon Dioxide.}}

  \acro{ECDI}{Electron cyclotron drift instability\acroextra{. Another name of the {EDI} present in {HET}.}}
  \acro{EDI}{$E\times B$ electron drift instability\acroextra{. Another name of the {ECDI} present in {HET}.}}
  \acro{DR}{Dispersion Relation \acroextra{. Relation between the wave number and complex frequency for waves in plasmas.}}
  
  \acro{IAW}{Ion Acoustic Wave}
  \acro{NWC}{Near-Wall Conductivity\acroextra{. Increased cross-field transport due to electron-wall collision and electron emissions from the wall.}}
  \acro{SEE}{Secondary Electron Emission\acroextra{. Electron emission from a wall due to an energetic impact of a primary electron.}}
  \acro{FT}{Fourier Transform}
  \acro{FFT}{Fast Fourier Transform}
  \acro{DFT}{Discrete Fourier Transform }
  \acro{BC}{Boundary Conditions }
  \acro{EP}{Electric Propulsion\acroextra{. Propulsion engines using the electric energy, instead of the chemical energy. }}
  \acro{CNES}{Centre National d'Etude Spatial\acroextra{. The French space agency}}
  \acro{ML}{Laboratory Model}
  \acro{EEDF}{Electron Energy Distribution Function }
  \acro{SCL}{Space Charge Limit}
  \acro{RMS}{Root Mean Square}
  \acro{RF}{Radio Frequency}
  \acro{LEO}{Low Earth Orbit \acroextra{\\ The LEO corresponds to orbits of altitude lower than 2000 km. Mostly used for Earth observation, and the ISS. }}
  \acro{GEO}{GEostationary Orbit \acroextra{\\Corresponds to the Orbit that follows the Earth rotation. Used mainly for telecommunication (Like the French Canal (former Canalsat)), it lies at 36000km from the Earth.}}
\end{acronym}
