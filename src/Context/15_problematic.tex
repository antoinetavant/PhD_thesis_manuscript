% !TEX root=/home/tavant/these/manuscript/src/manuscript.tex

\section{EP Industrial challenges}
\label{sec-challenges}
% \addcontentsline{toc}{section}{EP Industrial challenges}

Several challenges are currently tackled in the \ac{EP} industry.
The most prominent are listed by \citet{samukawa2012}\string:
\begin{enumerate}
  \item Performance improvement\string: efficiency, lifetime, and cost-effectiveness.
   Lifetime is an important issue and is limited by electrode or wall erosion.
   The lifetime of an electric thruster must be larger than 10 000 h of (reliable) operation.
   \item  Design of more versatile thrusters, i.e. able to operate at different combinations of thrust and propellant velocity.
   \item  Extension of the domain of operation to lower power ($\mu$N to 10 mN thrust range) for microsatellites or accurate attitude control.
   \item  Extension to higher power for orbit raising of telecommunication satellites (several tens of kW) and    interplanetary missions (100 kW and more).
   \item Extension of EP to low-altitude spacecraft\string: there is an increasing interest in civil and military satellites flying  at altitudes around 100 km where the drag is significant and must be continuously compensated.
\end{enumerate}

\ac{HET} technology has the potential to answer many of these challenges.
For instance, the lifetime is approached with wall-less and magnetically shielded configuration.
Versatility is tackled with dual-mode \ac{HET} configuration \citep{boniface2017}, low power thruster is attained with $\mu$-thrusters \citep{lascombes2018}, and so forth.
However, the development of \ac{HET}s is slow and expensive. 
A better physical understanding of the processes governing \ac{HET}s is needed in order to reduce the cost and the delays of the thrusters development.
This is the objective of the current collaboration between Safran Aircraft Engines and \ac{LPP}.
