% !TEX root=/home/tavant/these/manuscript/src/manuscript.tex

\section*{Problematic}
\label{sec-problematic}
\addcontentsline{toc}{section}{Problematic}

We have seen in the previous sections that the \ac{HET}s have been studied and used since several decades.
However, several challenges are currently tackled in the \ac{EP} industry, as the most prominent listed by \citet{samukawa2012} :
\begin{enumerate}
  \item Performance improvement: efficiency, lifetime and cost-effectiveness.
   Lifetime is an important issue and is limited by electrode or wall erosion.
   Lifetime of an electric thruster must be larger than 10 000 h of (reliable) operation.
   \item  Design of more versatile thrusters, i.e. able to operate at different combinations of thrust/propellant velocity.
   \item  Extension of domain of operation to lower power ($\mu$N to 10 mN thrust range) for microsatellites or very precise attitude control.
   \item  Extension to higher power for orbit raising of telecommunication satellites (several tens of kW) and    interplanetary missions (100 kW and more).
   \item Extension of EP to low-altitude spacecraft: there is an increasing interest in civil and military spacecraft flying  at altitudes around 100 km where the drag is significant and must be constantly compensated.
\end{enumerate}

The \ac{HET} technology has the potential to answer many of these challenges.
Lifetime is approached with wall-less and magnetically fielded configuration.
Versatility is tackled with dual-mode \ac{HET} configuration \citep{boniface2017}, low power thruster is attained with $\mu$-thursters \citep{lascombes2018}, and so forth.

Unfortunately, the developed of \ac{HET}s remains principally empirical. 
Two physical phenomena are still unclear, and block the understanding of \ac{HET}s \citep{samukawa2012,adamovich2017}:
\begin{itemize}
  \item the electron transport,
  \item the plasma-wall interaction.
\end{itemize}

In parallel to sophisticated diagnostics, efforts in the development of kinetic simulations is pursued.
Indeed, the electron transport is affected by instabilities that can only be described by kinetic models \citep{adam2008a,lafleur2016a}
In addition, the plasma-wall interaction is also affected by kinetic effects, both concerning the electron emission induced by electron impact \citep{barral2003a,raitses2011,sydorenko2006} and the wall erosion by ion impact sputtering.

Relatively few simulation codes highly parallelized have been developed, that could allow parametric studies.
But the ever increasing computational power available allows bigger simulations to be conducted, with for instance \ac{3D} simulation recently performed \citep{fubiani2018,taccogna2012}, even if a scaling was needed to obtain the results under a reasonable time.

The objective of the work performed during this thesis was to better understand the inner physics of the \ac{HET} discharge, more precisely the electron transport and the plasma-wall interaction. 
in order to answer this question, a kinetic simulation code, highly parallelized, is developed and used, and order to pin-down the main mechanisms, and propose macroscopic model by the mean of parametric studies.

\vspace{1em}
In Chapter 1, we introduce the simulation model.
Chapter 2 presents the results of a parametric study investigating the wall effect PIC.
In chapter 3, we revisit the sheath model in order to explain he simulation results.
The content of Chapter 4 and 5 are not yet fixed.
