% !TEX root=/home/tavant/these/manuscript/src/manuscript.tex

\section*{Problem statement and outline of the thesis}
\label{sec-problematic}
\addcontentsline{toc}{section}{Problematic of the thesis}

My thesis takes part of the collaboration between Safran Aircraft Engines and the Laboratory of Plasma Physics, which objective is to study the fundamental physics governing the \ac{HET}, in the optic to accelerate the developments of the next generations of thrusters.
I mainly focused on the electron transport and the plasma-wall interaction,
both aspects requiring the use of kinetic tools.

Indeed, the electron transport is affected by instabilities that can only be described by kinetic models \citep{adam2008a,lafleur2016a}.
Furthermore, the plasma-wall interaction is also affected by kinetic effects, both concerning the electron emission induced by electron impact \citep{barral2003a,raitses2011,sydorenko2006} and the wall erosion by ion impact sputtering.
Relatively few simulation codes highly parallelized have been developed, that could allow parametric studies.
But the ever-increasing computational power available allows bigger simulations to be conducted.
Thus, a significant part of my work involves the development of a highly efficient \ac{PIC} simulation code, with all of the technical difficulties related to it.
The simulation code is then used to proceed to several parametric studies, that I used to derive reliable low-dimensional models that could be used to derive new engineering development tools.


\vspace{1em}
In Chapter 1, we introduce the primary simulation model, as well as the fundamental theories of the plasma-wall interaction and the electron transport.
Chapter 2 presents the results of a parametric study investigating the wall effect.
In Chapters 3 and 4, we modify the sheath model in order to reproduce the \ac{PIC} simulation results.
Chapter 3 focuses on a simplified \ac{1D} simulation to study the electron state law, while Chapter 4 continues the same model by including the secondary electron emission.
Chapter 5 focus on the azimuthal instability observed in the simulation and compares it to the dispersion relation.
While the majority of the work studied the \ac{2D} radial and azimuthal simulation, we finish by study the radial direction in a \ac{2D} axial and azimuthal simulation.