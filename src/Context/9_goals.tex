% !TEX root=/home/tavant/these/manuscript/src/manuscript.tex

\section*{Problematic of the thesis}
\label{sec-problematic}
\addcontentsline{toc}{section}{Problematic of the thesis}

My thesis takes part in the collaboration between Safran Aircraft Engines and the Laboratory of Plasma Physics, which objective is to study the fundamental physics governing the \ac{HET}, in the optic to accelerate the developments of the next generations of thrusters.
I mainly focused on the the electron transport and the plasma-wall interaction,
both aspects requiring the use of kinetic tools.

Indeed, the electron transport is affected by instabilities that can only be described by kinetic models \citep{adam2008a,lafleur2016a}.
In addition, the plasma-wall interaction is also affected by kinetic effects, both concerning the electron emission induced by electron impact \citep{barral2003a,raitses2011,sydorenko2006} and the wall erosion by ion impact sputtering.
Relatively few simulation codes highly parallelized have been developed, that could allow parametric studies.
But the ever increasing computational power available allows bigger simulations to be conducted.
Thus, a significant part of my work involves the development of a highly efficient \ac{PIC} simulation code, with all of the technical difficulties related to it.
The simulation code is then used to proceed to several parametric studies, that I used to derived reliable low-dimensional models.


\vspace{1em}
In Chapter 1, we introduce the main simulation model.
Chapter 2 presents the results of a parametric study investigating the wall effect.
In chapters 3 and 4, we modify the sheath model in order to reproduce the simulation results.
Chapter 5 (and maybe 6) focus on the azimuthal instability.
