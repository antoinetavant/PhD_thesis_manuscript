% !TEX root=/home/tavant/these/manuscript/src/manuscript.tex


\section*{\ac{HET} research and development with Safran}
\label{sec-poseidon}
\addcontentsline{toc}{section}{\ac{HET} research and development with Safran}



Safran Aircraft Engines has been collaborating with \ac{LPP} since 2014, starting with the Ph.D. thesis of Viven Croes \citep{croes2017}.
During this first three years, a \ac{2D} \ac{PIC} code has been developed simulating the radial and azimuthal directions of a \ac{HET}.
Azimuthal instabilities have been observed in \citet{croes2017a}, and the effects of alternative propellants have been investigated in \citet{croes2018}.

From this fruitful collaboration, an ANR (Agence National de la Recherche) industrial chair {\sc Poseidon} for  "uture Plasma thrusters for LOw earth orbit SatEllIte propulsiON systems", Grant No. ANR-16-CHIN-0003-01 , has been created.
Its objective is to develop novels methods to reduce the development time and cost of the next \ac{EP} systems.
Both experiments and simulations are being developed to unlock the barriers of \ac{HET} development.
The {\sc Poseidon} chair is linked to the current development of a  low power \ac{HET} at Safran, the \PPS X00, which nominal operating point is of the order of 500W.
The scientific part of the chair is leaded by \ac{LPP}, while an unstructured \ac{3D} simulation code is developed by the CERFACS, in Toulouse.
Safran leads the engineering development and experimental investigation.

At the begining of my thesis, I participated to the development of a \ac{ML} of the \PPS X00.
The objectives of the \PPS X00-\ac{ML}  is to represent the physics of the \PPS X00 while allowing parametric studies of the main parameters of a \ac{HET}, as the geometry, the magnetic field topology, or the wall material.
The \PPS X00-\ac{ML} has successfully showed its usefulness, as the first tests allow to obtain state of the art performances \citep{vaudolon2018}.
My work at Safran showed us that the development of \ac{HET} is still currently driven by experiments, because numerical tools are not yet predictive.
Simulations are helpful for the engineers to have some insights for the thruster behaviour, but cannot be used for development with confidence.
However, experiments are costly and time-consuming.
They also are prone to delays in the conception schedule, and reduce innovation as designers take less risks.

The lack of numerical tools for \ac{HET}  comes from some keys physical phenomena that need to be better understood.


\section*{Key phenomena of \ac{HET}s}
\addcontentsline{toc}{section}{Key phenomena of \ac{HET}s}

Even though \ac{HET} have been studied and used for more than 40 years, some key phenomena are still ill-understood, and then a \emph{trial and errors} method is used for the development by manufacturers.
These key phenomena are
\begin{itemize}
  \item the electron axial transport toward the anode
  \item the plasma-surface interaction 
  \item the wall erosion
  \item the propellant nature
\end{itemize}


\paragraph{The electron axial transport}  through the magnetic barrier has been measured much higher than the expected value from the classical collisional theory by \citet{meezan2001}.
Different phenomena have been proposed to explain the origin of this \emph{anomalous} transport.
Two phenomena are supposed to be mainly responsible for this enhanced mobility\string: the azimuthal instability and the near wall mobility due to electron emission.
A significant part of the work of this Ph.D. thesis concerns the quantitative comparison of the relative importance of the two phenomena.

\paragraph{Plasma-wall interaction} concerns the phenomena that depends of the wall material and affect the discharge.
Indeed, depending on the nature of the wall, the discharge behaviour can vary \citep{gascon2003}.
The phenomenon responsible for this observation is the electron emission, as an electron reaching the wall will have a different probability to emit one or more electrons for different materials.
This affects the particle and power balance of the plasma, hence affecting the sheath and plasma characteristics.
Ion induced electron emission is much less likely to happen as the ions have a small energy of impact.

\paragraph{Wall erosion} is due to sputtering of the ceramic induced by ions impact.
This large erosion is the main limitation of \ac{HET}s lifetime.
While most aspects of the erosion are well understood, we observe the apparition or  patterns with a typical scale of the order of the millimeter in the eroded surfaces that could affect the performances.
The origin, and implication, of these erosion striations stays an open question.

\paragraph{The propellant nature} affects the chemistry of the plasma in the thruster.
Because of its high mass and low ionization energy, xenon has been used since the beginning of \ac{HET}. 
However, it is very costly.
The cheaper, but less effective, propellant of choice is krypton, that started to be used.
Iodine could also be interesting, as it can be stored at room temperature in a solid state.
The impact of the propellant mass and chemistry is not yet clear, and slows down the use of alternative propellants.

\vspace{1em}
The objectives of my thesis, in the context of the {\sc Poseidon} chair focuses on the two first points -- the plasma wall interaction and the electron mobility -- and how they can influence each-other.
\inlinenote{Of the Three point, if I talk about erosions ?}





