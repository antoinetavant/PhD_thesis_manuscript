% !TEX root=/home/tavant/these/manuscript/src/manuscript.tex


\section*{Instabilities present in the \ac{HET} }
\label{sec-physics}
\addcontentsline{toc}{section}{Instabilities present in the \ac{HET}}

The \ac{HET} are subject to numerous plasma oscillations, over a large range of frequencies \citep{boeuf2017,choueiri2001}.
The most important one are\string:
\begin{enumerate}
  \item Low frequency (10-20\,\kilo\hertz) ionization oscillations, usually refereed as breathing mode,
  \item Azimuthal low frequency rotating spokes, also in the \kilo\hertz range,
  \item Axial ion transit time oscillations, of the order of 100-500 \kilo\hertz,
  \item Azimuthal fast oscillation, of frequency of the order of the ion plasma frequency.
\end{enumerate} 

\paragraph{1. Breathing mode\\}
The breathing mode is relatively well understood \citep{boeuf1998,barral2009,hara2014}.
Indeed, a simple predator-prey model of two equation is enough to qualitatively obtain the observed behaviour.
It is related to the idea that when the ionization is strong, the neutral atom density decreases, reducing the ionization.
Hence, the plasma density decreases, allowing the neutral density to rise again until the ionization growth back.

\paragraph{2. Rotating spokes\\}
Experimental measurements with segmented anode \citep{ellison2012,mcdonald2011} seems to indicate that the rotating spokes are present in the anode region.
Their physical origins are less understood, as they were first attributed to ionization \citep{janes1966}, but were later related to Simon-Hoh instability and observed in \ac{PIC} simulation, even with neglecting ionization \citep{carlsson2018}.
However, in recent experiments, the presence of spokes did not seem to affect the \ac{HET} performances \citep{boeuf2018}.

\paragraph{3. Transit time instability\\}
Transit time instabilities have been predicted and observed in analytical and numerical models \citep{barral2005,boeuf2018}.
Experimental studies of these instabilities are rather scarce and it is only recently that time-resolved Laser Induced Fluorescence measurements of the local ion velocity distribution function have confirmed the presence of this instability in a Hall thruster \citep{vaudolon2015}.
This oscillation can reduce the thruster performances by increasing the overlap between the acceleration and ionization regions \citep{boeuf2018}.

\paragraph{4. High frequency azimuthal oscillations\\}
As introduced in \cref{sec-HET}, these instabilities rises from the electron $E\times B$ drift.
They were first observed in \ac{PIC} simulation \citep{adam2004,ducrocq2006,adam2008a,heron2013} before being observed by electron Thomson scattering \citep{tsikata2009a,tsikata2009,tsikata2013}.
These instability are better described in \cref{ch-5}, especially in \cref{sec-DR-kinetic}.
