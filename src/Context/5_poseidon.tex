% !TEX root=/home/tavant/these/manuscript/src/manuscript.tex


\section*{\ac{HET} research and development}
\label{sec-poseidon}
\addcontentsline{toc}{section}{\ac{HET} research and development}

Safran Aircraft Engines has been collaborating with \ac{LPP} since 2014, starting with the Ph.D. thesis of Viven Croes \citep{croes2017}.
During this first three years, a \ac{2D} \ac{PIC} code has been developed simulating the radial and azimuthal directions of a \ac{HET}.
Azimuthal instabilities have been observed in \citet{croes2017a}, and the effects of alternative propellants have been investigated in \citet{croes2018}.

From this fruitful collaboration, an ANR (Agence National de la Recherche) industrial chair {\sc Poseidon} for  "future Plasma thrusters for LOw earth orbit SatEllIte propulsiON systems", Grant No. ANR-16-CHIN-0003-01, has been created.
Its objective is to develop novels methods to reduce the development time and cost of the next \ac{EP} systems.
Both experiments and simulations are being developed to unlock the barriers of \ac{HET} development.
The {\sc Poseidon} chair is linked to the current development of a  low power \ac{HET} at Safran, the \PPS X00, which nominal operating point is of the order of 600W.
The scientific part of the chair is leaded by \ac{LPP}, while an unstructured \ac{3D} simulation code is developed by the CERFACS, in Toulouse.
Safran leads the engineering development and experimental investigation.

At the begining of my thesis, I participated to the development of a \ac{ML} of the \PPS X00.
The objectives of the \PPS X00-\ac{ML}  is to represent the physics of the \PPS X00 while allowing parametric studies of the main parameters of a \ac{HET}, as the geometry, the magnetic field topology, or the wall material.
The \PPS X00-\ac{ML} has successfully showed its usefulness, as the first tests allow to obtain state of the art performances \citep{vaudolon2018}.
My work at Safran showed us that the development of \ac{HET} is still currently driven by experiments, because numerical tools are not yet predictive.
Simulations can be helpful to engineers in order to obtain some insights for the thruster behaviour, but cannot be used with confidence for development.
On the other hand, experiments are costly and time-consuming.
They also are prone to delays in the conception schedule, and reduce innovation as designers take less risks.

The lack of numerical tools for \ac{HET} comes from some physical phenomena that need to be better understood, even though \ac{HET} have been studied and used for more than 40 years.
% Hence a \emph{trial and errors} method is currently used by manufacturers for the development of the \ac{HET}.
These key phenomena are \citep{samukawa2012,adamovich2017}
\begin{itemize}
  \item the electron transport,
  \item the plasma-surface interaction ,
  \item the wall erosion,
  \item the propellant nature.
\end{itemize}

\vspace{1em}
The propellant nature impact principally two things\string: the ion mass and the ionization energy.
Because of its high mass and low ionization energy, xenon has been used since the beginning of \ac{HET}. 
However, it is very costly, as it is mostly extracted from fin air with cryogenic distillation.
However, air is composed in average of $\sn{9}{-6}\%$ of xenon \citep{earthfacs}.
The cheaper, but less effective, propellant of choice is krypton, which have recently started to be used.
Iodine could also be interesting, as it can be stored at room temperature in a solid state.
The impact of the propellant mass and chemistry is not yet clear, and slows down the use of alternative propellants on already design systems.

\vspace{1em}
The objectives of my thesis in the context of the {\sc Poseidon} chair focuses on the two first points -- the plasma wall interaction and the electron mobility -- and how they can influence each-other.
I also studied the wall erosion, but this work is classified, and will not be disclosed in this manuscript.





