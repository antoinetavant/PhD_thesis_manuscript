% !TEX root=/home/tavant/these/manuscript/src/manuscript.tex


\section*{Poseidon chair}
\label{sec-poseidon}

The work of this thesis as taken part of the POSEIDON project.

A few words here.


Safran Aircraft engines is collaborating with \ac{LPP} since 2014, with the Ph.D. thesis of Viven Croes \citep{croes2017}.
During this first three years, a \ac{2D} \ac{PIC} code has been developed simulating the radial and azimuthal directions of a \ac{HET}.
Azimuthal instabilities have been observed in \citet{croes2017a}, and the effects of alternative propellants have been investigated in \citet{croes2018}.
From this fruitful collaboration, a ANR (Agence National de la Recherche) chair has been created.
Its objective is to develop novels methods to reduce the development time and cost of the next \ac{EP} systems.
Both experiments and simulations are being developed to unlock the barriers of \ac{HET} development.

The POSEIDON chair is linked to the current development of a \ac{HET} of low power at Safran, the \PPS X00, which nominal functioning point is of the order of 500W.
The scientific part of the chair is leaded by \ac{LPP}.
An unstructured \ac{3D} simulation code is developed by the CERFACS, in Toulouse.
Safran leads the engineering development and experimental investigation.

During my Ph.D., I participated to the development of a \ac{ML} of the \PPS X00.
The objectives of the \PPS X00-\ac{ML}  is to represent the physics of the \PPS X00 while allowing parametric study of the main parameters of a \ac{HET}, as the geometry, the magnetic field topology, or the wall material.
The \PPS X00-\ac{ML} has successfully showed its usefulness, as the first tests allow to obtain state of the art performances \citep{vaudolon2018}.

