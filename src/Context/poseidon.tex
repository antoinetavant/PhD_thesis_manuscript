% !TEX root=/home/tavant/these/manuscript/src/manuscript.tex


\section*{\ac{HET} research and development with Safran}
\label{sec-poseidon}
\addcontentsline{toc}{section}{\ac{HET} research and development with Safran}



Safran Aircraft engines has been collaborating with \ac{LPP} since 2014, starting with the Ph.D. thesis of Viven Croes \citep{croes2017}.
During this first three years, a \ac{2D} \ac{PIC} code has been developed simulating the radial and azimuthal directions of a \ac{HET}.
Azimuthal instabilities have been observed in \citet{croes2017a}, and the effects of alternative propellants have been investigated in \citet{croes2018}.

From this fruitful collaboration, an ANR (Agence National de la Recherche) chair has been created.
Its objective is to develop novels methods to reduce the development time and cost of the next \ac{EP} systems.
Both experiments and simulations are being developed to unlock the barriers of \ac{HET} development.

The POSEIDON chair is linked to the current development of a  low power \ac{HET} at Safran, the \PPS X00, which nominal operating point is of the order of 500W.
The scientific part of the chair is leaded by \ac{LPP}, while an unstructured \ac{3D} simulation code is developed by the CERFACS, in Toulouse.
Safran leads the engineering development and experimental investigation.

During my Ph.D., I participated to the development of a \ac{ML} of the \PPS X00.
The objectives of the \PPS X00-\ac{ML}  is to represent the physics of the \PPS X00 while allowing parametric study of the main parameters of a \ac{HET}, as the geometry, the magnetic field topology, or the wall material.
The \PPS X00-\ac{ML} has successfully showed its usefulness, as the first tests allow to obtain state of the art performances \citep{vaudolon2018}.

\inlinenote{ Here, add a few words on the difficulties see at Safran ? }

\section*{Key phenomena of \ac{HET}s}
\addcontentsline{toc}{section}{Key phenomena of \ac{HET}s}

Even though \ac{HET} have been studied and used for more than 40 years, some keep phenomena are still ill-understood, making the development of \ac{HET} by manufacturers to a \emph{ trial and error} method.
These key phenomena are
\begin{itemize}
  \item the electron transport toward the anode
  \item the plasma-surface interaction 
  \item the wall erosion, limiting the lift-time of \ac{HET}s
  \item the propellant nature impact on the performances
\end{itemize}

My work mainly focuses on the two first point, and how they can influence each-other.

\paragraph{The electron axial transport}  through the magnetic barrier has been measured much higher than the expected value from the classical collisional theory in \citet{meezan2001}.
Different phenomena have been proposed to explain the origin of this \emph{anomalous} transport, mainly the azimuthal instability and the near wall mobility due to electron emission.
A significant part of the work of the thesis concerns the quantitative comparison of the relative influence of the two phenomena.

\paragraph{Plasma-wall interaction} concerns the phenomena that depends of the wall material and affect the discharge, mainly the secondary electron emission.
Indeed, depending the the nature of the wall, an electron reaching the wall will have a different probability to emit one or more electrons.
This affect the particle and power balance of the plasma, hence affecting the sheath and plasma characteristics.
Ion induce electron emission is much less likely to happen as the ions have a small energy of impact.

\paragraph{Wall erosion} is due to sputtering of the ceramic due to the ion impact.
This large erosion is the main limitation of \ac{HET}s lifetime.
The erosion of the wall presents two stages,
\begin{itemize}
  \item the first happens at the 
  \item the first happens at the 
\end{itemize}




