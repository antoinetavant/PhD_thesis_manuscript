% !TEX root=/home/tavant/these/manuscript/src/manuscript.tex

\section{Conclusion on the 2D PIC axial-azimuthal simulations}

In this chapter, we have investigated the impact of the radial direction in a \ac{PIC} simulation that simulates the axial and azimuthal directions.
We propose a model in order to introduce in a \ac{2D} azimuthal-axial simulation the effects of the radial direction.
More precisely, we model the losses of particle and energy, corresponding to fully absorbing walls without electron remission.
The model imposes pseudo-local flux neutrality at the wall, reproducing the effect of a floating sheath.

We investigate two different test-cases have. The first one, proposed by \citet{boeuf2018}, uses a forced ionization source term.
This approach allows us to suppress the ionization instability, also named breathing mode.
The second one was proposed by \citet{coche2014}. 
The ionization term is self-consistently computed using the \ac{MCC} algorithm.
However, in order to reduce the computation load, the authors introduced a scaling of the vacuum permittivity, allowing us to to resolve the breathing mode.

\vspace{1em}
The model of radial loss has successfully shown its ability to reduce the plasma density and electron temperature in the two test-cases.
In the test-cases of Coche, the breathing mode is affected by the radial losses. 
Its frequency is reduced, and its growth rate is increased.

Using the test-cases of Boeuf, we were able the obtain a steady-state for three cases\string: one without radial losses, and two with losses using a radial length $L_R=4\,\centi\meter$ and $L_R=2\,\centi\meter$.
We observe that the radial losses reduce the electron density slightly, and the axial electric field is almost not affected.
On the other hand, the electron temperature is significantly reduced in the three directions, and not only in the radial direction.
Consequently, the amplitude of the azimuthal electric field is also reduced, diminishing the axial electron transport.

In addition, we see that the wave characteristics are affected.
More precisely, in the downstream region, a low-frequency and large wavelength is present when no radial losses are modeled.
The amplitude of this large wave is reduced for $L_R=4\,\centi\meter$, and disappears for $L_R=2\,\centi\meter$.
The origin of the instability modification is not clear.
It might be due to the change of the plasma density and the electron temperature.
The numerical noise, induced by the radial loss algorithm, could also affect the wave.
More investigation must be conducted.

\vspace{1em}
We observed that the electrons are less isotropic than seen in the radial-azimuthal simulation, even in the presence of electron-neutral scattering.
We noted that in the radial simulation, the mean radial electron Joule heating is not zero in the plasma bulk.
This radial heating seem to be due to the instability, that presents radial structures at saturations.
This joule heating is absent of the \ac{2D} axial-azimuthal simulation.
Thus, as long as the radial heating of the electrons is not clearly understood, the plasma-wall interaction cannot be realistically modeled in such low-dimensional simulations.


\inlinenote{super pessimiste pour une derniere phrase de chapitre 6!!!!
Hauts les coeurs!!! ;-)
essaie de reformuler cette phrase sans negation!!!}