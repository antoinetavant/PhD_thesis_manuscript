% !TEX root=/home/tavant/these/manuscript/src/manuscript.tex

\section{Conclusion on the axial-azimuthal simulations}

In this chapter, we have investigated the impact of the radial direction in a \ac{PIC} simulation simulating the axial and azimuthal direction.

We have proposed a model in order to introduce in a \ac{2D} azimuthal-axial simulation of the effects of the radial direction.
More precisely, we started by model the losses, corresponding to fully absorbing walls without electron remission.
The model imposes a pseudo-local flux neutrality at the wall, reproducing the effect of a floating sheath.

Two different model of simulation has been investigated. The first one, proposed by \citet{boeuf2018}, uses a forced ionization source term. this allows to suppress the ionization instability, also named breathing mode.
The second model was proposed by \citet{coche2014}. 
The ionization term is selfconsistently computed using the \ac{MCC} algorithm.
However, in order to reduce the computation load, the authors introduced a scaling of the vacuum permittivity, allowing the easily resolve the breathing mode.

\vspace{1em}
The model as successfully shown is ability to reduce the plasma density and electron temperature in the 2 models.
In the model of Coche, the breathing mode is affected by the radial losses. 
Its frequency is reduced, and its growth rate is increase.

Using the model of Boeuf, we are able the obtain a steady state for three cases, one without radial losses, and two with losses, using a radial length $L_R=4\,\centi\meter$ and $L_R=2\,\centi\meter$.
We observe that the radial losses reduces slightly the electron density, and the axial electric field is almost not affected.
On the other hand, the electron temperature is significantly reduced in the three direction, and not only in the radial direction.
Consequently, the amplitude of the azimuthal electric field is also reduces, reducing the axial electron transport.

In addition, we see that the wave characteristics are affected.
More precisely, in the downstream region, a low-frequency and large wavelength is present when no radial losses is model.
The amplitude of this large wave is reduced for $L_R=4\,\centi\meter$, and disappear for $\L_R=2\,\centi\meter$.

The origin of the instability modification is not clear.
It might be due to the modification of the plasma density and the electron temperature.
It could also be due to a numerical noise, induced by the radial loss model.
More investigation should be conducted.

\vspace{1em}
We observed that the electrons are less isotropic than observed in the radial-azimuthal simulation, even in the presence of electron-neutral scattering.
We observed that in the radial simulation, the mean radial electron Joule heating is not zero in the bulk.
This radial heating would seems to be due to the instability, that presents radial structure at saturations.
This joule heating is absent of the \ac{2D} axial-azimuthal simulation.
Thus, as long as the radial heating of the electron is not clearly understood, the plasma-wall interaction cannot be realistically modeled in such low-dimensional simulations.
