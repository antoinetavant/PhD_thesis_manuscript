% !TEX root=/home/tavant/these/manuscript/src/manuscript.tex

\section{PIC simulation results: Boeuf model}


We start by presenting the results of the simulation case of Boeuf, since it is simpler to analyse.
Three different cases are presented.
The first is the usual case, without the effects of the radial direction.
It is expected to return the same results as obtained in \citet{boeuf2018}.
The two other cases model the radial effect of the radial direction.
Two values of the radial length are used\string: $L_R=4\,\centi\meter$ and $2\,\centi\meter$.

\subsection{Boeuf model: Temporal evolution} \label{subsec-temp_boeuf}

\Cref{fig-boeuf-temporal} present the temporal evolution of the mean plasma density and temperature.
We can see that when the radial direction is modeled, the density and the temperature is reduces as expected when increasing the losses.

\begin{figure}[hbtp]
  \centering
  \begin{tabular}{cc}
    \subfigure{Boeuf_ne_temporal}{a}{20,20} &
    \subfigure{Boeuf_Te_temporal}{b}{20,15} \\
  \end{tabular}
  \caption{({\bf a}) temporal evolution of the mean plasma density and  ({\bf b})  temporal evolution of the mean  electron temperature obtained with 3 different radial model for the simulation case of Boeuf. }
  \label{fig-boeuf-temporal}
\end{figure}

We can see on \cref{fig-boeuf-temporal}.{\bf b} that the radial temperature stays constant when the radial direction is not modeled.
this is due to the fact that the collisions are not modeled in the simulations, so that there is not momentum and energy transfer possible from the axial and azimuthal direction, and the radial direction.

However, we know from the radial-azimuthal simulations that there is a transfer, resulting in a plasma less anisotropic than observed here.
Consequently, the radial losses are far from the one expected in reality.


\subsection{Boeuf model: Axial profiles} \label{subsec-axial_boeuf}

\Cref{fig-boeuf_axialone,fig-boeuf_axialtwo}  show the axial profile at steady-state of several plasma quantities.
The variables are averaged in the azimuthal direction and in time between $t=8$ and $t=10\,\micro\second$.
The results  obtained with 3 different radial model for the simulation case of Boeuf are overlaid.
\cref{fig-boeuf_axialone}.{\bf a} shows the axial electric field $E_z$.
The differences in $E_z$ are small, but we can still distinguish that the amplitude of $E_z$ is reduced with the increased radial losses.
The electron density $n_e$ showed in \cref{fig-boeuf_axialone}.{\bf b} is almost not affected, except in for $z>1\,\centi\meter$, where the electron losses at the wall can be seen.

\begin{figure}[hbtp]
  \centering
  \begin{tabular}{cc}
    \subfigure{Boeuf_electric_field}{a}{30,22} &
    \subfigure{Boeuf_ne_axial}{b}{30,22} \\
  \end{tabular}
  \caption{Averaged axial profile at steady state of ({\bf a}) the axial electric field $E_z$, ({\bf b}) the mean plasma density obtained with 3 different radial model for the simulation case of Boeuf. }
  \label{fig-boeuf_axialone}
\end{figure}

On the other hand, the axial electron density current $J_{e, z}$ (\cref{fig-boeuf_axialtwo}.{\bf a}) and the electron temperature (\cref{fig-boeuf_axialtwo}.{\bf b}) are significantly reduced when the radial losses are modeled.
We can see a factor of two on $J_{e, z}$ between the case $L_R=2\,\centi\meter$ and the case without losses.
However, we have seen that the electron density is almost unaffected.
This means that the electron axial velocity is reduced, because of the radial losses.

\begin{figure}[hbtp]
  \centering
  \begin{tabular}{cc}
    \subfigure{Boeuf_Je_axial}{a}{30,22} &
    \subfigure{Boeuf_Te_axial}{b}{25,80} \\
  \end{tabular}
  \caption{Averaged axial profile at steady state of ({\bf a})  the axial electron current density $J_{e, z}$, and ({\bf b}) the  electron temperature obtained with 3 different radial model for the simulation case of Boeuf. }
  \label{fig-boeuf_axialtwo}
\end{figure}


We see that the electron transport is highly affected.
This is expected to be due to the reduced electron temperature, which reduces the amplitude of the instability at saturation.
It is confirmed by the values presented in \Cref{fig-boeuf-instability}.
\cref{fig-boeuf-instability} shows on the left the average standard deviation of the azimuthal electric field.
It represent the amplitude of the oscillation.
We can see that the amplitude of the instability is significantly affected by the electron radial losses.

\begin{figure}[hbtp]
  \centering
  \includegraphics[width=\textwidth]{Boeuf_instability_characteristics}
  \caption{Axial profile of the characteristics of the instability, (left) average of the standard deviation of the azimuthal electric field, (right) electron-ion friction force calculated by the correlation between $n_e$ and $\Te$.    }
  \label{fig-boeuf-instability}
\end{figure}

On the right panel of \cref{fig-boeuf-instability} we can see the correlation term
\begin{equation} \label{eq-rei}
  R_{ei} = < n_e E_{\theta} >
\end{equation}
responsible for the electron axial enhanced transport.
Again, we can see a decrease on the amplitude of $R_{ei}$ because of the reduced oscillation amplitude.


\subsection{Boeuf model with collisions} \label{subsec-MCC_boeuf}
% Script on Juno (CH-6_Boeuf)

We have seen that because there is no collision in the case of Boeuf, there is no energy transfer toward the radial direction.
Thus, we investigate here the impact of a neutral on the electron anisotropy by modeling the collision with the \ac{MCC} algorithm.

The xenon neutrals are injected at the anode side with density of $n_g=1e17\,meter$${-3}$, an axial velocity $v_g = 200 \,\meter\per\second$ and a temperature $T_g=640\,\kelvin$.
We model the neutral flow with the continuity and momentum conservation equation, while keeping their temperature constant.
The ionization source term used to sustains the plasma is a loss term in the continuity equation.
\Cref{fig-boeuf-neutrals} shows the neutral density and velocity obtained at steady-state.

\begin{figure}[hbtp]
  \centering
  \begin{tabular}{cc}
    \subfigure{boeuf_MCC_ng}{a}{20,20} &
    \subfigure{boeuf_MCC_vg}{b}{20,15} \\
  \end{tabular}
  \caption{Axial profile at steady-state ($t=14\,\micro\second$) of ({\bf a}) the neutral density and  ({\bf b})  the neutral axial velocity, for  simulation case of Boeuf with the electron-neutral scattering. }
  \label{fig-boeuf-neutrals}
\end{figure}
We can see in \cref{fig-boeuf-neutrals} that the neutrals are significantly depleted because of the forced the ionization source term.
The neutral density gradient accelerates the neutrals in the axial direction by a factor of two, which reduces even more the neutral density.
thus, the electron-neutral scattering will be important on the anode side of the chamber.

\begin{figure}[hbtp]
  \centering
  \begin{tabular}{cc}
    \subfigure{boeuf_mean_Te}{a}{20,20} &
    \subfigure{boeuf_mean_Tez_profile_MCC}{b}{20,15} \\
  \end{tabular}
  \caption{({\bf a}) temporal evolution of the electron kinetic energy and  ({\bf b}) axial profile of the electron temperature obtained for the simulation case of Boeuf with the electron-neutral scattering. }
  \label{fig-boeuf-temporal}
\end{figure}

\Cref{fig-boeuf-temporal} shows the temporal evolution of the mean electron kinetic energy of the left, and on the right it shows the axial profile of the electron temperatures, decomposed on the three directions.
We can see now that they is a small transfer of energy between the radial direction and the others.
In particular, close the the anode, the radial and axial temperatures decreases below the initial temperature $\Te_{, inj}=5\,\volt$.
In contrast, at the maximum of the magnetic field, the radial energy increases to $\Te_R=7\,\volt$.
However, the anisotropy stays significant, compared to the  radial-azimuthal simulations (\cref{ch-2}).

\subsection{Radial electron heating} \label{subsec-radial-heating}
The large anisotropy observed in \Cref{fig-boeuf-temporal} compared to the results of \cref{ch-2} might be due to the difference on neutral density.
However, the collisions are usually neglected in \ac{HET}s compared to the instability.
Hence, the plasma fluctuations may be responsible for the electron radial heating.

The electron are power gain is the Joule heating
\begin{equation} \label{eq-epower}
    \vect{P_{\rm J}} = \vect{J_e} \cdot \vect{E},
\end{equation}
with $\vect{J_e}$ the electron density current and $\vect{E}$ the electric field.
\Cref{fig-epower_radial} shows the mean Joule heating in the radial direction $\bar{P_{\rm J, R}} = < J_{e, R} E_R >$, and the product of the mean quantities $< J_{e, R}>  < E_R >$.

\begin{figure}[hbtp]
  \centering
  \includegraphics[width=\textwidth]{R_joule_heating_one}
  \caption{Electron current and radial electric field in the Radial-azimuthal simulations }
  \label{fig-epower_radialone}
\end{figure}


\begin{figure}[hbtp]
  \centering
  \includegraphics[width=\defaultwidth]{R_joule_heating_two}
  \caption{Electron power gain in the Radial-azimuthal }
  \label{fig-epower_radial}
\end{figure}

We see that the mean Joule heating $\bar{P_{\rm J, R}}$ is not zero in the center of the simulation, while the product of the mean is zero.
Indeed, the mean electric field in the bulk is almost zero.
This means that there is an energy transfer to the radial direction of the electrons due to the correlation between $\vect{J_e}$ and $\vect{E}$.
For this energy transfer to be present, the radial direction needs to be resolved.

A similar radial heating has been observed by \citet{heron2013}.
The authors observe no heating when the instability was only perpendicular to the magnetic field, as it is in a \ac{1D} or a \ac{2D} axial-azimuthal simulation.
However, when the direction parallel to the magnetic field is resolved, the electrons are heated, but the physical mechanism remains unclear.
In \citet{janhunen}, the authors observe a similar radial electron heating, but due to the presence of a \ac{MTSI}.
Has discussed previously, we do not observe the \ac{MTSI} in our simulation, meaning that it much comes from another mechanism.


\subsection{Electron azimuthal drift velocity} \label{subsec-drift}

We showed in \cref{ch-5} that the instability growth rate is proportional to the electron azimuthal drift.
The the radial-azimuthal simulation, the drift was only due to the $\vect{E} \times \vect{B}$ drift
\begin{equation} \label{eq-exbdrift}
  u_{E \times B} = - \frac{E_z}{B_r}
\end{equation}

\Cref{fig-Jetheta_sum} shows the axial profile of the azimuthal electron mean velocity $u_{e, \theta}$ at steady-state ($t = 14\,\micro\second$) for the case without the radial losses modeled.
The drift velocity $u_{E \times B}$ is also shown.
We can see that the electron velocity is no more equal to the $u_{E \times B}$ drift.
Instead, we saw in \cref{fig-boeuf_axial} that the electron density presents a large axial gradient.
This leads to the diamagnetic drift
$$u_{\rm Dia}=\frac{\nabla_z (n_e {\rm T}_{e,z})}{n_e B_r}.$$
The value of the diamagnetic can be seen in \cref{fig-Jetheta_sum}.


\begin{figure}[hbtp]
  \centering
  \includegraphics[width=\defaultwidth]{Boeuf_Je_x_axial_one}
  \caption{Axial profile of the electron azimuthal velocity, the $\vect{E} \times \vect{B}$ drift velocity and the diamagnetic velocity and some of the $\vect{E} \times \vect{B}$ and diamagnetic velocity at steady state for the simulation case of Boeuf without radial losses.}
  \label{fig-Jetheta_sum}
\end{figure}

We can see that $u_{\rm Dia}$ is of the same order of magnitude that $u_{E \times B}$, but of the opposite sign.
Moreover, we see that we have 
$$ u_{e, \theta} =   u_{E \times B} + u_{\rm Dia}$$
everywhere in the simulation domain.
\Cref{fig-Jetheta} shows the values of $ u_{e, \theta},   u_{E \times B}$, and $u_{\rm Dia}$ for the three cases.
We can see that the magnitude of $u_{e, \theta} $ decreases when the radial losses are present.
However, the amplitude of both $u_{\rm Dia}$ and $u_{E \times B}$ decreases as well.

 
\begin{figure}[hbtp]
  \centering
  \includegraphics[width=\defaultwidth]{Boeuf_Je_x_axial}
  \caption{Axial profile of the electron azimuthal velocity, the $\vect{E} \times \vect{B}$ drift velocity and the diamagnetic velocity at steady state with 3 different radial model for the simulation case of Boeuf.}
  \label{fig-Jetheta}
\end{figure}

