% !TEX root=/home/tavant/these/manuscript/src/manuscript.tex

\section{Bidimentionnal simulation of an Hall Effect Thruster}


\subsection{The Hall effect Thruster }

The \ac{HET} is an electrostatic electrical propulsion system accelerating ions by the mean of an imposed voltage difference.

We can summarize its composition by four parts:
\begin{enumerate}
  \item The annular chamber.
  \item The injecting anode
  \item The cathode
  \item The magnetic circuit
\end{enumerate}

\paragraph{The chamber} has an annular shape.
It is open closed at the anode side, and kept open at the other side.
The walls are usually constituted by a ceramic, usually \ac{BNSiO2}.
The material needs to be resistant to erosion by ion impact sputtering.
But changing the material is also known to affects the discharge behaviour.
The usually supposed phenomena for this impact is the secondary electron emission yield that is a function of the material nature.


\paragraph{The anode} is at the bottom of the chamber.
The anode voltage is imposed to a few hundred volts.
Usually, the neutral gas injection is made by the anode itself.
The mass flow rate is of the order of a flew mg/s.

\paragraph{The cathode} is outside of the chamber.
It is grounded, and injects electrons for two reasons:
\begin{itemize}
  \item most of the electrons ($\sim 90 \%$) are used to neutralize the ion flux, for both allowing the ions to leave the thruster and avoid charging of the spacecraft.
  \item some of the electrons are attracted by the anode, hence entering the chamber and allowing the plasma discharge and switch and remain on.
\end{itemize}
