
% !TEX root=/home/tavant/these/manuscript/src/manuscript.tex

\section{Elements of the 2D PIC-MCC simulations}

\subsection{Principe of the PIC simulations}

The \ac{PIC} simulation models the movement of particles on a fixed grid.

The grid is used to compute the electric field, in the electrostatic approximation by solving the Poisson equation \vref{eq-poisson}, or the Maxwell equations

\begin{equation}
  \label{eq-poisson}
  \nabla \phi = - \frac{\rho}{\epsilon_0}
\end{equation}

where $\phi$ is the electric potential, $\rho$ is the charge density, and $\epsilon_0$ the vacuum permittivity.

The charge density $\rho$ is computed by depositing the particle over the cells, using the usual Cloud-in-cell model \cite{birdsall1991}.

The particles moves following the Loren'z forces \cref{eq-Lor}.

\begin{equation}
  \label{eq-Lor}
  m \vec{a} = q E + q \vec{v} \times \vec{B}
\end{equation}
with $m$ and $q$ the particle mass and electric charge respectively.
The numerical particles folowed in the simulations correspond to $q_f$ physical particles, with
\begin{equation}
  q_f = \frac{n V}{N}
\end{equation}
with $n$ the particle density, $V$ the volume of a cell, and $N_{\rm pc}$ the number of numerical particle in a cell.
A large enough number of particle is needed in order to obtain physical results.
Indeed, insufficient number of particles leads to numerical heating \cite{ueda1994}.
Usually, a minimum of 100 particles per cell are used, but recent results seem to encourage to use more particle \cite{janhunen2018}

\subsection{Monte Carlo collisions}

In \ac{PIC} simulations, collisions between charged and neutral particles can be modeled by binary collision, but this approach is computationally costly.
Instead, a Monte-Carlo algorithm can be used \cite{vahedi1995}.
This approach is very efficient, and allow scattering, momentum transfer and ionization to be consistently model.
