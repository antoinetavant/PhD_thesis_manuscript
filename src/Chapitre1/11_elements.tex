% !TEX root=/home/tavant/these/manuscript/src/manuscript.tex

\section{Elements of the 2D PIC-MCC simulations}
  \label{sec-elements}
  \subsection{Principe of the PIC simulations}

    The \ac{PIC} simulation models particles moving freely on a grid.
    The grid is used to compute the electric field, in the electrostatic approximation by solving the Poisson equation

    \begin{equation}
      \label{eq-poisson}
      \Delta \phi = - \frac{\rho}{\epsilon_0}
    \end{equation}

    where $\phi$ is the electric potential, $\rho$ is the charge density, and $\epsilon_0$ the vacuum permittivity.
    If the electrostatic approximation is not correct, one need to solve the Maxwell equations.

    The particles move following the Lorenz forces
    \begin{equation}
      \label{eq-Lor}
      m \vec{a} = q \vect{E} + q \vec{v} \times \vec{B}
    \end{equation}
    with $m$ and $q$ the particle mass and electric charge respectively.
    The numerical particles followed in the simulations correspond to $q_f$ physical particles, with
    \begin{equation}
      q_f = \frac{n V}{\Npc}
    \end{equation}
    with $n$ the particle density, $V$ the volume of a cell, and $\Npc$ the number of numerical particles in a cell.
    A large enough number of particles is needed in order to obtain physical results.
    Indeed, insufficient number of particles leads to numerical heating \cite{ueda1994}.
    Usually, a minimum of 100 particles per cell are used, but recent results seem to encourage to use more particles \cite{janhunen2018}.

  \subsection{Monte Carlo collisions}

    In \ac{PIC} simulations, collisions between charged and neutral particles can be modeled by binary collision, but this approach is computationally costly.
    Instead, a Monte-Carlo algorithm can be used \cite{vahedi1995}.
    This approach is very efficient, and allow scattering, momentum transfer and ionization to be consistently modeled.
    The propellant used in \ac{HET} is \ac{Xe}.
    The cross sections used for modeling \ac{Xe} or other gases collisions are taken from the {\sc LXCat} database project \cite{LXCat_web,pancheshnyi2012}.
    Except if otherwise stated, the elastic, inelastic scattering and ionization reactions listed in \cref{tab-reactXe} are used.
    The cross section values are summarised in \cref{fig-xexsection}.

    \begin{table}[hbtp]
      \ra{1.3}
      \centering
      \caption{Reactions for xenon used in the PIC simulations}
      \label{tab-reactXe}
      \begin{tabular}{@{}lll@{}}  \toprule
        Reaction & Threshold & Reference\\ \midrule
        {\it Elastic scattering} & &\\
        e + Xe = e + Xe   & --   & \cite{Lxcat_Xe,Lxcat_Xe2} \\
        {\it Excitation} & &\\
        e + Xe = e + Xe$^*$   & 8.315eV   & \cite{Lxcat_Xe,Lxcat_Xe2} \\
        e + Xe = e + Xe$^*$   & 9.447eV   & \cite{Lxcat_Xe,Lxcat_Xe2} \\
        e + Xe = e + Xe$^*$   & 9.917eV   & \cite{Lxcat_Xe,Lxcat_Xe2} \\
        e + Xe = e + Xe$^*$   & 11.7eV    & \cite{Lxcat_Xe,Lxcat_Xe2} \\
        {\it Ionization} & &\\
        e + Xe = e + Xe$^+$   & 12.13eV   & \cite{Lxcat_Xe,Lxcat_Xe2} \\
        \bottomrule
      \end{tabular}
    \end{table}



    \begin{figure}[hbtp]
      \centering
      \includegraphics[width=\defaultwidth]{figure/xenon_cross_section.pdf}
      \caption{Cross section values used in the Monte Carlo procedure \cite{Lxcat_Xe,Lxcat_Xe2}.}
      \label{fig-xexsection}
    \end{figure}


\section{Numerical implementation of the Particle in cell simulation}

  \LPPic is an explicit electrostatic \ac{PIC}-\ac{MCC} simulation code.
  Every time-steps, the simulation loop presented in \cref{fig-picloop} is computed.
  The different steps constituting the PIC-loop are described in the next subsections.
  \begin{figure}[hbtp]
    \centering
    \smartdiagramset{circular distance=4.5cm,
    module minimum width=3.5cm,
    text width=3cm,
    arrow tip=to}
    \smartdiagram[circular diagram:clockwise]{Particle Motion ,Boundary,Collision,Density weighting, Poisson Equation, Field weighting}
    \caption{\ac{PIC}-\ac{MCC} loop executed every time steps.}
    \label{fig-picloop}
  \end{figure}


  \subsection{Data used}
    In \ac{PIC} simulations, there are two kinds of data used\string:
    \begin{itemize}
      \item Particles (electron, ions, neutrals can be followed as well but not in \LPPic)
      \item Mesh, also named fields (densities, electric and magnetic fields, and so on)
    \end{itemize}

    \paragraph{Particles\\}
    For each particles, are known its position $\vec{x}$ and its velocity $\vec{v}$.
    In most \ac{PIC}-\ac{MCC} simulations, the 3 directions of the velocity vector are followed in order to take into account scattering.
    It is abbreviated as \acs{3V}.
    The particle positions and velocity are not discretize, except to the numerical floating-point precision.

    \paragraph{Fields\\}
    The fields are defined at the center of each cell of the mesh.
    The charge density $\rho$ is computed by depositing the particle on the mesh, using the Cloud-in-cell model \cite{birdsall1991}.
    The electric field at the position of the particle is also obtained by bi-linear interpolation.
    The mesh dimension defines the dimension of the simulation.
    It is usual to find \acs{1D}\acs{3V} or \acs{2D}\acs{3V} \ac{PIC} simulations, for particles with 3 directions on the velocity but 1 (or 2) dimensions in space.

    \subsection{Particle pusher}
    The interaction of the movement equation \cref{eq-Lor} is different for magnetized and non-magnetized particles.
    For non-magnetized particles, we use the leapfrog scheme \cite{birdsall1991}
    \begin{align}\label{eq-leapfrog}
      \vect{v}^t &= \vect{v}^{t-1} + \frac{q}{m} \vect{E} \dt, \\
      \vect{x}^t &= \vect{x}^{t-1} + \vect{v}^t \dt,
    \end{align}
    with the superscript $t$ designing the time step, $q$ and $m$ the particle electric charge and mass, $\vect{E}$ the electric field at the particle position, and \dt the time step duration.

    It is important to note that the leapfrog induces a shift of $\frac{\dt}{2}$ between the position and the velocity, as illustrated in \cref{fig-leapfrog}.
    \begin{figure}[hbtp]
      \centering
      \includegraphics[width=\defaultwidth]{leapfrog.png}
      \caption{Illustration of the shift between the particle velocity and position.}
      \label{fig-leapfrog}
    \end{figure}
    This shift can leads to erroneous diagnostics when computing moments of the particles distribution.
    For instance, the mean velocity of an ensemble of $N$ particles at the instant $t$ is computed as\string:
    \begin{equation} \label{eq-meanv}
      \mean{\vect{v}}^t = \frac{1}{N} \sum_i^N \lp \vect{v_i}^t + \frac{q}{m} \vect{E_i} \frac{\dt}{2} \rp.
    \end{equation}
    Other moments like the mean energy or heat flux follow the same correction.
    We can see that the error between $\mean{\vect{v}}$ defined above and
    $$ \tilde{\vect{v}} = \frac{1}{N} \sum_i^N  \vect{v_i}^t $$
    is
    $$ \mean{\vect{v}} - \tilde{\vect{v}} =\frac{q \dt}{2 m}  \frac{1}{N}  \sum_i^N  \vect{E_i} .$$
    Hence, the error in the diagnostic is larger in the region of large electric field (as in the sheaths).

    \paragraph{Magnetized particles}
    For magnetized particles, we use a modification of the leapfrog algorithm proposed by Boris \cite{boris1970}.
    It corresponds to an operator splitting between the electrostatic acceleration and the magnetic rotation.
    This splitting is described below\string:

    \begin{enumerate}
      \item accelerate the particle during $\frac{\dt}{2}$\string: $\vect{v}^{t-\frac{\dt}{2}} = \vect{v}^{t-1} + \frac{q}{m} \vect{E} \frac{\dt}{2}$
      \item rotate the particle velocity with the magnetic field
      \item accelerate the particle during $\frac{\dt}{2}$\string: $\vect{v}^t = \vect{v}^{t-\frac{\dt}{2}} + \frac{q}{m} \vect{E} \frac{\dt}{2}$
    \end{enumerate}


  \subsection{Poisson equation solver}
  \label{subsec-poissonintro}

    In order to compute the electron field due to the particle charge density, the Poisson equation \cref{eq-poisson}  need to be discretized over the mesh.
    We can directly discretize the differential operator by using the finite volume approach over a cell of the mesh.
    The formal discretization is developed in \cref{sec-diel}, for the special case of taking into account the presence of dielectric boundaries.

    In \ac{1D}, the obtained linear system is tridiagnonal.
    It can be solved directly using {\sc Thomas} algorithm, which simply stores the Gauss elimination's coefficient.
    In \ac{2D}, the linear system is pentadiagonal.
    A direct solver, like the $LU$ decomposition, would require a large amount a memory to store the factorisation matrices.
    On the other hand, as the time step is usually small in \ac{PIC} simulation, we expect the plasma potential $\phi$ not to change rapidly.
    Hence, an iterative solver using the previous solution as initial guess seems more reasonable from both the memory storage and the computational time.

    