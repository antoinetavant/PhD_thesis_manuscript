% !TEX root=/home/tavant/these/manuscript/src/manuscript.tex



\section{Conclusion}
  \label{sec-conclusion_ch1}
  

  In order to study the plasma wall interaction in an \ac{HET}, we developed a bi-dimensional simulation code using \ac{PIC}-\ac{MCC} modeling.
  As the electrons drift azimuthally due to the $E \times B$ configuration, the \ac{ECDI} rises, enhancing the cross-field transport of the electron towards the anode.
  The walls closing the chamber in the radial direction are also important for the discharge behavior.
  Hence, in order to compare the interaction between these phenomena, we simulate the radial-azimuthal domain.
  
  A special care have been taken during the developement of \LPPic, the \ac{2D}-\ac{3V} \ac{PIC}-\ac{MCC} simulation code, concerning
  \begin{itemize}
    \item the modeling of the axial convection, in order to model the energy losses and so attain a steady-state,
    \item the modeling of the radial boundary with the dielectric layer included in the simulation domain.
  \end{itemize}
  
  Several theories have also been given in order to better understand the \ac{PIC} simulation results, especially concerning the electron cross-field mobility and the sheath model in presence of electron emission.