% !TEX root=/home/tavant/these/manuscript/src/manuscript.tex

\chapter{Scalability tests}
\label{an-scalability}

In this annexe, we give more information on the scalability tests conducted.
We conducted both a weak and a strong scalability tests.
We recall that the strong scalability keeps the load constant while increasing the computational performance, here the number of CPU.
The weak scalability keeps the load to the number of CPU ratio constant.
However, it does not correspond to the need we had, which is to find the optimal number of CPU for a given task.
Beside, the results of the weak scalability were less clear to analyze.
Hence, we discarded them.

The parameters of the strong scalability test used are given in \cref{parameters-strong-scalability}.
They are close to the parameter used in production, presented in \cref{parameters}.
The smaller test-case uses the same settings, except for the domain which is divided by a factor of 4.
The dielectric boundary conditions are not modeled (no \ac{SEE}, no dielectric layer).
The Inputs, set-up, and Outputs are not taken into account in the study of the performances.
However, the usual diagnostics (mean density, fluxes, temperature, at boundaries and in the plasma) are computed.
The performances are averaged over the 1000 first time steps of the simulations.
Hence, the load is well balanced between the CPU. 


\begin{table}[htbp] %PIC parameters
     \centering
     \ra{1.3}
     \caption{\label{parameters-strong-scalability} Operating and numerical parameters used in the strong scalability large test-case.}
     \begin{tabular}{@{}r c c c@{}} 
        \toprule
        {\bf Physical Parameter} & notation & Value & Unit \\
        \midrule
        Gas & & Xenon & - \\
        Domain dimensions & $L_{x} \times L_{y}$ & $1.0 \times 1.023$ & [cm$^2$] \\
        Radial magnetic field & $B_{0}$                    & $200$                 & [{G}] \\
        Axial electric field & $E_{0}$                    & $2 \times 10^{4}$     & [{Vm}$^{-1}$] \\
        Mean plasma density & $n_{0}$                    & $1 \times 10^{17}$    & [{m}$^{-3}$] \\
        Initial electron temperature & $\Te_{,0}  $               & $1.0$                 & [{V}] \\
        Initial ion temperature & $T_{i,0}   $               & $0.05$                 & [{V}] \\
        Neutral gas pressure & $P_{n}     $               & $0.1$                 & [{mTorr}] \\
        Neutral gas temperature & $T_{n}     $               & $300$                 & [{K}] \\
        Neutral gas density & $n_{g}     $               & $3.22 \times 10^{18}$ & [{m}$^{-3}$]\\
        \midrule
        {\bf Simulation Parameter} &  &   &  \\
        
        Time step & $\Delta t  $                      & $4 \times 10^{-13}$ & [{s}] \\
        Cell size & $\Delta x = \Delta y = \Delta z $ & $5 \times 10^{-7}$  & [{m}] \\
        Number of particles per cell & $N/NG      $                      & $150$                & [{part/cell}] \\
        Number of particles & $N      $                      & $2 \times 614099924$                & [{particles}] \\
        
        Number of time steps & $N_t$ & 1000 & --\\
        
        
        \bottomrule
     \end{tabular}
  \end{table}
  
  \begin{table}[hbtp]
  \ra{1.3}
    \centering
    \caption{Modified parameter for the strong-scalability small test-case. The other parameters are given in \cref{parameters-strong-scalability}. }
    \label{tab-peremeters-small}
    \begin{tabular}{@{}r c c c@{}} 
      \toprule
      {\bf Physical Parameter} & notation & Value & Unit \\
      \midrule
      Domain dimensions & $L_{x} \times L_{y}$ & $1.0 \times 1.023$ & [cm$^2$] \\
      \midrule
      {\bf Simulation Parameter} &  &   &  \\
      Cell size & $\Delta x = \Delta y = \Delta z $ & $1 \times 10^{-6}$  & [{m}] \\
    \bottomrule
    \end{tabular}
  \end{table}
  
  An example of the relative importance of each module of the simulation over a time step is given in \cref{tab-example_perf}.
  Each module is called 1000 times, except for the outputs function, that writes the diagnostics to the disk, which is called only once.
  Its duration is divided by the number of time steps, in order to compare it to the other modules.
  To give an order of magnitude, taking $t=10.08\,\second$ for one time-step corresponds to an average of $8.2\,\nano\second$ per particle.
  
  \begin{table}[hbtp]
  \ra{1.3}
    \centering
    \caption{Performances of the large test-case (parameters of \cref{parameters-strong-scalability}) when using 96 CPUs, average over 1000 time steps. }
    \label{tab-example_perf}
    \begin{tabular}{@{}r c c@{}} 
      \toprule
    Module  & Duration one time-step [s]  & Percentage \\
    \midrule
    Total & 10.08 & 100 \\
    Diagnostics particle to mesh & 6.03 & 59.9 \\
    Particle motion & 2.88 & 28.6 \\
    Monte Carlo Collision & 0.729 & 7.31 \\
    Outputs & 0.287 & 2.84 \\
    Poisson solver & 0.075 & 0.75 \\
    \bottomrule
    \end{tabular}
  \end{table}
  
