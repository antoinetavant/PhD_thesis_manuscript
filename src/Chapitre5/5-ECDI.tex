% !TEX root=/home/tavant/these/manuscript/src/manuscript.tex




\chapter{Azimuthal instability}
\label{ch-5}

\begin{Chabstract}
  
As briefly mentioned in \cref{ch-1}, the $E \times B$ configuration of the \ac{HET} give rise to azimuthal instabilities.
This aspect has been neglected in \cref{ch-2,ch-3}, except by they consequences on the axial electron transport.
While these instabilities are the subject of numerous studied, they remain unclear.

Using the results of the \ac{PIC} simulations, we propose new insights for the understanding on the instability, hence the electron cross-field transport.
In particular, we develop a relation dispersion solver that uses the velocity distribution function measured in the simulations.
Then, we compare the simulation instability characteristics with the dispersion relations.
A special care is taken with the boundary condition and the instability non-linear saturation. 
\end{Chabstract}

% 
% 
% {\bf V. analyse of the instability } 30 pages
% \begin{zzz}
%   \begin{itemize}
% \item Instability dispersion relation
% \item Kinetic solver for Ion and ECDI using VDFs
% \item Impact of the difference w/r Maxwellian
% \item Wall Boundary condition, 3D dispertion relation vs 1D and 2D relations
% \item Linear stage as ECDI and Saturation toward IAW
% \end{itemize}
% \end{zzz}

\minitoc


The presence of azimuthal instabilities in the Hall effect thrusters has first been showed with numerical simulations by \citet{adam2004}.
However, their nature is not yet clear.
The investigation of the instabilities observed in the \ac{2D} \ac{PIC} simulations in the subject of this chapter.

In \cref{sec-PIC-ECDI}, we present the oscillations observed in the \ac{PIC} simulations.
After that, we derive the dispersion relation with no hypothesis used concerning the particle distribution functions in \cref{sec-DR-kinetic}, and we present in \cref{sec-DR-solver} a numerical algorithm that solves the dispersion relation using the distribution function measured in the \ac{PIC} simulations.
The oscillations observed in the simulation are compared in \cref{sec-DR-results} to the results of the dispersion relation.
To finish with; the impact of the radial boundary condition is investigated in \cref{sec-DR-BC}.

% !TEX root=/home/tavant/these/manuscript/src/manuscript.tex

\section{Instability in the \ac{PIC} simulation}
  \label{sec-PIC-ECDI}
  As briefly mentioned in \cref{ch-1}, the $E \times B$ configuration of the \ac{HET} make azimuthal instabilities to rise.
  This aspect has been neglected in \cref{ch-2,ch-3}, expect when talking about the axial electron transport.
  While these instabilities are the subject of numerous studied, they remain unclear.
  
  Using the results of the \ac{PIC} simulation, we propose some new insights for the understanding on the instability, hence the electron cross-field transport. 
  \inlinenote{Maybe this part in the chapter abstract...?}
  
  We present in this section the results obtained for one case.
  However, we observe the same results when varying the  different physical parameters.
  The parameters of the simulations are presented in \cref{tab-evdfpicparams}.
  The radial boundary condition include a dielectric layer, of width $L_{diel}$.
  No electron emission is modeled, and the convection is modeled with the new model (see \cref{sec-noiselessresults}).

  \begin{table}[hbtp]
  \ra{1.3}
    \centering
    \caption{Parameters of the \ac{PIC} simulations}
    \label{tab-evdfpicparams}
    \begin{tabular}{@{}l l l @{}} \toprule
    Parameter    &   Value   &  Unit  \\ \midrule
    $L_r\times L_{\theta}\times L_z$   & $1\times 0.26\times 0.5$ & cm \\
    $n_e = n_i$  & $\sn{3}{17}$ & \per\meter\cubed \\
    $B_r$  & 0.02 & T \\
    $E_z$  & \sn{2}{4} & \volt\per\meter \\
    $L_{diel}$ & 3 & \milli\meter  \\
    \bottomrule
    \end{tabular}
  \end{table}

  \subsection{Azimuthal electric field}
  
  \Cref{fig-2DcutEx} shows the temporal evolution of the azimuthal electric field as a function of the azimuthal position, measured at the center of the radial direction.
  We clearly see the instability growth up to the saturation around $T=1\,\micro\second$.
  Then, we see in addition to the fast oscillation a slower modulation of the amplitude. 
  \begin{figure}[hbtp]
    \centering
    \includegraphics[width=\defaultwidth]{electric_field_cut2D}
    \caption{Temporal evolution of the azimuthal electric field as a function of the azimuthal position.}
    \label{fig-2DcutEx}
  \end{figure}

  \Cref{fig-FFT_ex} shows the frequency spectrum of the azimuthal electric field presented in \cref{fig-2DcutEx} computed via \ac{FFT} in the stationary state ($t > 1.2\,\micro\second$).
  The spectrum have been average in the azimuthal direction, in order to reduce the noise.
  The theoretical frequency $f_{\rm theo} = \frac{\opi}{\pi \sqrt{6} }$ is given. {\bf REF}
  We can see a very good agreement between $f_{\rm theo}$ and the maximum of the frequency spectrum.
  \begin{figure}[hbtp]
    \centering
    \includegraphics[width=\defaultwidth]{spectrum_frequency}
    \caption{Frequency spectrum of the azimuthal electric field, averaged in the azimuthal direction. The black line is the theoretical frequency.}
    \label{fig-FFT_ex}
  \end{figure}
  
  \subsection{Temporal evolution} \label{subsec-temp}
  We have seen in the previous section that after a growing phase, the instability saturate but oscillates slowly around.
  This section analyse these temporal characteristics.
  
  \Cref{fig-Ezstd_time} shows the temporal evolution of the characteristics of the electrostatic wave.
  As the wave is not monochromatic (i.e. it is the sum of multiple waves), we display both the maximum of the electric field $\max(E_{\theta})$, and its standard deviation $\stdE$.
  In the case of a monochromatic wave, the would have 
  \[ \stdE = \frac{\max(E_{\theta})}{\sqrt{2}}.  \]
  
  \begin{figure}[hbtp]
    \centering
    \includegraphics[width=\defaultwidth]{Temporal_E_theta.pdf}
    \caption{Temporal evolution of the maximum and the standard deviation of the azimuthal electric field, in log scale.}
    \label{fig-Ezstd_time}
  \end{figure}
  
  We can see in \cref{fig-Ezstd_time} that during the first microsecond, we observe an exponential growth, corresponding to a constant growth rate.
  A linear fit in log scale give $\gamma_{PIC} \simeq 0.07 \opi$ during the linear phase.
  After $t=1\,\micro\second$, the amplitude of the electric field oscillates around a mean value, with a period of the order of $T_{NL}=1.5 \,\micro\second$.
  Several phenomena are candidate to the modulation observed.
  
  
  \paragraph{Ion transit time\\}
    The ion are injected at the anode, and are accelerated by the uniform axial electric field $E_z$.
    The transit time of the ions in the axial direction $T_t$  is the time needed for the ions to travel $L_z$
    \begin{equation} \label{eq-transittime}
      T_{t} = \sqrt{\frac{2 m_i L_z}{e E_z}} \simeq 0.8 \mu s.
    \end{equation}
    
    The transit time is of the good  order of magnitude, but $T_{NL}$ is twice bigger, still.
    
  \paragraph{Particle trapping and bouncing\\}
    A common raison for wave saturation is the ion trapping. 
    It as been observed in both \ac{1D} \citet{lafleur2016a} and \ac{2D} \citep{croes2017a}.
    
    Hence, the low frequency modulation could be due to particle bouncing \citep{belmont2013}.
    However, the bouncing time scale is 
    \begin{equation} \label{eq-TB}
      T_{B} = 2 \pi \sqrt{\frac{m_i}{e k \max(E_{\theta})} } \simeq 0.5 \,\micro\second,
    \end{equation}
    which is 3 times smaller than $T_{NL}$.
    Even though in \citet{belmont2013}, the authors say that, when the amplitude of the electric field is large (as it is the case here), the bouncing time scale increases due to non-linear phenomenon (the particle trajectory is not harmonic any more), we cannot conclude here that this is the origin of the low frequency modulation.

  
  \paragraph{Ion trapping oscillation\\}
    The wave saturating due to ion trapping have an amplitude of \citep{boeuf2018}
     \begin{equation} \label{eq-iontropempl}
       \stdE \leq \frac{\Te}{12 \lde}.
     \end{equation}
    
    Defining the wave energy  density
    \begin{equation} \label{eq-waveE}
      \epsilon_{\rm wave} = \frac{\epsilon_0}{2} \stdE^2
    \end{equation}
    and the electron thermal energy density
    \begin{equation} \label{eq-thE}
      \epsilon_{\rm th} = \frac{3}{2} e n_e \Te
    \end{equation}
    
    We find the criteria for the ion trapping
    \begin{equation} \label{eq-criteriaIT}
      432 \epsilon_{\rm wave} \leq \epsilon_{\rm th}.
    \end{equation}
    
    \Cref{fig-tempITcrit} shows the temporal evolution of the electron thermal energy $\epsilon_{\rm th}$ and the wave energy density $\epsilon_{\rm wave}$, scaled by the factor 432.
    We can see that $\epsilon_{\rm th}$ is relatively constant.
    However, $\epsilon_{\rm wave}$  oscillates significantly, and passes above and below $\epsilon_{\rm th}$.
    
    \begin{figure}[hbtp]
      \centering
      % \includegraphics[width=\defaultwidth]{Ion_Trapping_criter}
      \includegraphics[width=\defaultwidth]{Ion_Trapping_criter_bis}
      \caption{Temporal evolution of the wave energy density (scaled) compared to the thermal energy density.}
      \label{fig-tempITcrit}
    \end{figure}
    
    In \cref{fig-tempITcrit}, as expected the period of the oscillations of $\epsilon_{\rm wave}$ is $T_{NL} = 1.5\,\micro\second$.
    But here, we can see that when the criteria \cref{eq-criteriaIT} is fulfilled, the temporal derivative of the wave energy density increases, which means that the wave growing rate is positive.
    However, some time after the moment in violated the criteria is violated ($\tau = 0.40 \pm 0.07 \,\micro\second$ on the four oscillations), the wave abruptly stop rising but decreases instead.
    This delay between the time the ions should be trapped and the time the wave stop growing in most certainly due to the ion inertia, as we have $\tau \sim T_B$.
    
    A complementary results is presented in \cref{subsec-VDFIAW} but solving the dispersion relations what we present in the next section.
    
      
% !TEX root=/home/tavant/these/manuscript/src/manuscript.tex

\section{Dispersion relation of the instabilities}
  \label{sec-DR-kinetic}
  
  
  The dispersion relations are obtained by coupling the particle dynamics with the electric fields.
  In the case of the kinetic electrostatic dispersion relation, we couple the Vlasov equation with the Poisson equation.
  
  In our \ac{2D} geometrie, we can neglect all of the gradients.
  Are imposed a fixed axial electric field $\vect{E} = E_0 \vect{e_z}$ and a fixed radial magnetic field $\vect{B} = B_0 \vect{e_r}$
  The electrons are drifting in the azimuthal direction due to the $E\times B$ drift, hence
  \[ \vect{u_{e}} = \frac{\vect{E} \times \vect{B}}{\norm{\vect{B}}^2} = \frac{E_0}{B_0}  \vect{e_{\theta}}.    \]
  The ions are accelerated in the axial direction, $ \vect{u_i} = u_i  \vect{e_z}$.
  
  The perturbations are waves of the form \[ \exp \lp i \vect{k} \cdot \vect{x} - i \omega t  \rp, \]
  with $\vect{x}$ the position vector, the oscillation wave vector $\vect{k}$ is real but its frequency $\omega = \omega_r + i \gamma$ can be complex, where $\gamma$ is the growth rate of the oscillations. 
  
  \vspace{1em}
  The dispersion relation has been studied by \citet{ducrocq2006} in the case of cold ions and Maxwellian electrons in a \ac{2D} geometry.
  A numerical algorithm has been proposed by \citet{cavalier2013} and compared to experimental measurements.
  In \citet{lafleur2016}, the authors added the ion drift velocity to the dispersion relation.
  
  The \ac{ECDI} dispersion relation obtained presents resonances at the cyclotron frequencies, which broadens when $k_r$ in increased.
  The limit when $k_r$ tends to large values is similar to an \ac{IAW}.
  This limit is usually used.
  
  Recently, \ac{2D} \ac{PIC} simulation on a large domain observed radial structures in the oscillations \citep{janhunen2018,hara2019a}.
  
  \citet{lafleur2018} got read of the Maxwellian electron hypothesis by using directly the electron distribution function measured in the \ac{PIC} simulations in the \ac{IAW} dispersion relation.
  He showed that the frequency of the oscillation is almost unperturbed, but the growth rate is significantly reduced, even when the ions were still supposed Maxwellian.
  
  Here, we propose to continue the investigation by solving the dispersion relation numerically with the electron and ion distribution function for both the \ac{IAW} and the \ac{ECDI}.
  
  
  \subsection{General dispersion relation}
  \label{sec-geneDR}
  
  We follow the development presented in \citet{ducrocq2006,cavalier2013,lafleur2016}.
  The plasma dielectric function is defined as
  \begin{equation} \label{eq-de}
    \hat\epsilon(\vect{k},\omega) = 1 - \sum_s \chi_s(\vect{k},\omega)
  \end{equation}
  
  where $\chi_s(\vect{k}, \omega)$ is the susceptibility of the species $s$.
  It is obtained by coupling the Poisson equation with the particles description.
  The dispersion relation is obtained by setting $  \hat\epsilon(\vect{k},\omega) =0$
  
  
  For the unmagnetized ions, supposing a Maxwellian distribution, the susceptibility is
  \begin{equation} \label{eq-}
    \chi_i(\vect{k},\omega) = \frac{\opi^2}{k^2 v^2_{th, i}} Z'\lp \frac{\omega - \vect{k} \cdot \vect{u_{i}}}{k v_{th, i}}  \rp
  \end{equation}
  where $\opi$ the ion plasma pulsation, $k=\norm{\vect{k}}$ and $\vect{u_i}$ is the mean velocity of the ions.
  The function $Z'$ is the derivative of the Fried and Conte function \citep{fried1961}
  \begin{equation} \label{eq-friedandConte}
    Z(\eta) = \frac{1}{\sqrt{\pi}} \int_{-\infty}^{\infty} \frac{\exp{(-t^2)}}{t - \eta} dt.
  \end{equation}
  We use here the Fried and Conte function because of the Maxwellian hypothesis.
  In \citet{xiehua-sheng2013}, the author proposes a numerical calculation of the susceptibility for a general distribution function
  \begin{equation} \label{eq-general}
    Z(\eta, f) = \int_{-\infty}^{\infty} \frac{f(t)}{t - \eta} dt,
  \end{equation}
  with $f$ the velocity distribution function to consider.
  For sake of brevity, the generalized dispersion function $Z(\eta,f)$ is noted $Z(\eta)$, and the Fried and Conte function is noted $Z_M(\eta)$.
  The derivative of $Z$ is
  \begin{equation} \label{eq-derivatives}
    Z'(\eta) = \int_{-\infty}^{\infty} \frac{\partial f(t) / \partial t}{t - \eta} dt,
  \end{equation}
  
  \vspace{1em}
  A general expression for the plasma dielectric function can be obtained for magnetized electrons by making use of the method of characteristics and is given by
  \begin{equation} \label{eq-drECDI}
    \begin{split}
    \hat\epsilon(\vect{k},\omega) =& 1 - \\
     &\frac{\opi^2}{k^2 v^2_{th, i}} Z'\lp \frac{\omega - \vect{k} \cdot \vect{u_{i}}}{k v_{th, i}}  \rp + \\
     & \frac{1}{k^2 \lde^2} \lb 1 + \lp  \frac{\omega - \vect{k} \cdot \vect{u_{e}}}{k v_{th, e}} \rp \sum_{n=-\infty}^{\infty} e^{- \beta} I_n(\beta) Z\lp  \frac{\omega - \vect{k} \cdot \vect{u_{e} - n \oce}}{k_{r} v_{th, e}} \rp  \rb,
  \end{split}
  \end{equation}
  where $I_n$ are the modified Bessel functions of the first kind, and 
  \begin{equation} \label{eq-beta}
    \beta = \frac{(k_{\theta}^2 + k_z^2) b^2_{th, e}}{ \oce^2}
  \end{equation}
  
  \Cref{eq-drECDI} is the dispersion relation for drifting magnetised electrons and unmagnetized ions.
  It will be used to study the \acf{ECDI}.
  


  \subsection{Modified Ion Accoustic Wave}
  \label{sucsec-IAW}
  \citet{lafleur2016} shows that after some assumptions -- mostly a drifting Maxwellian distributions and a small electron drift velocity compared to thermal speed -- \cref{eq-drECDI} can be solved to obtain
  \begin{equation} \label{eq-MIAW}
    \omega = \omega_r + i \gamma = \vect{k} \cdot \vect{u_{i}} \pm \frac{k c_s}{\sqrt{1 + k^2 \lde^2}} \pm i \sqrt{\frac{\pi m_e}{8 m_i}} \frac{\vect{k} \cdot \vect{u_{e}}}{( 1 + k^2 \lde^2)^{3/2}}.
  \end{equation}
  The above equation represent the modified ion-acoustic dispersion relation.
  
% !TEX root=/home/tavant/these/manuscript/src/manuscript.tex

\section{Solving the kinetic \ac{DR} for general distribution functions}
  \label{sec-DR-solver}
  
  In order to solve the dispersion relation $\hat\epsilon(\vect{k}, \omega) = 0$ of \cref{eq-drECDI}, we solve for every $\vect{k}$ the complex value of $\omega$ that minimise $\norm{\hat\epsilon}$.
  Hence, we first need to evaluate $\hat\epsilon(\vect{k}, \omega)$ for given arguments $\vect{k}, \omega$.
  Then, we can find the root.
  
  \subsection{Numerical determination of $\hat\epsilon$} \label{subsec-numepsilon}
  
  The numerical computation of $\norm{\hat\epsilon}$ is done using the method of \citet{xiehua-sheng2013} for computing $\tilde{Z}$ the numerical approximation of $Z$. 
  It uses the fact that $Z$ is an Hilbert transform of the distribution function, and that the Hilbert transform can be changed to a Fourier Transform, weighed by well chosen basis functions, so the \ac{FFT} algorithm can be used \citep{weideman1995}.
  
  An interesting point is that the evaluation of $\tilde{Z}(\eta)$ is done by a polynomial, which coefficients depends only on the distribution function $f$. 
  Hence, they need to be determined only once, and evaluating $\tilde{Z}(\eta)$ is relatively fast.
  
  \begin{figure}[hbtp]
    \centering
    \includegraphics[width=6in]{Validation_numericalZ.pdf}
    \caption{Comparison of the numerical evaluation of $\tilde{Z}_M(\eta)$ for a Maxwellian distribution function with the Fried and Conte function (from the \texttt{plasmapy} python package).  }
    \label{fig-numZ}
  \end{figure}
  \cref{fig-numZ} shows the comparison of the numerical evaluation of $\tilde{Z}(\eta)$ for a Maxwellian distribution function with the Fried and Conte function (from the `plasmapy` python package) for different values of the imaginary part of $\eta$.
  We can see that for $\Im(\eta)$ positive, null or slightly negative, the two functions gives exactly the same results.
  However, for larger negative values of $\Im(\eta)$, the two functions gives different results.
  
  This discrepancy between $Z_M$ and $\tilde{Z}_M$ for large negative imaginary argument is certainly do to the fact of the analytic expansion of $Z$ to the complex plan require the evaluation of the distribution function for complex velocities \citep{xiehua-sheng2013,weideman1995}.
  However, the discrete velocity distribution function measured in the \ac{PIC} simulation cannot be properly evaluated for complex velocities.
  On the other hand, we are only interested in instabilities, with positive growth rate.
  Hence, the discrepancy observed should not affect the conclusions of the study.
  


  \begin{figure}[hbtp]
    \centering
    \includegraphics[width=0.9\textwidth]{Validation_numericalZ_bis.png}
    \caption{Comparison of the numerical evaluation of $\hat\epsilon$ from \cref{eq-drECDI} for a Maxwellian distribution function with (left) the Fried and Conte function (from the \texttt{plasmapy} python package) and (right) the numerical $\tilde{Z}$, in logarithmic scale.  }
    \label{fig-numZbis}
  \end{figure}
  
  \Cref{fig-numZbis} presents the comparison of the calculation of $\hat\epsilon$ from \cref{eq-drECDI} for a Maxwellian distribution function with (left) the Fried and Conte function (from the `plasmapy` python package) and (right) the numerical $\tilde{Z}$.
  We can see that the root with the greater imaginary part, located close to (0.2, 0), is the same in both cases, but that other roots for large negative imaginary part are not similar.
  But as said previously, theses roots are not our concern, hence the dispersion relation should be well computed using $\tilde{Z}$.
  
  
  \subsection{Finding the root of $\hat\epsilon$}
  Now that we can compute $\hat\epsilon$, we can solve the dispersion relation.
  In order to find the root of $\hat\epsilon$, two methods have been tested.
  
  
  \paragraph{Exact root finding algorithm\\}
    The first approach has the advantage of finding all of the root in a given domain.
    It uses Cauchy's argument principle in order to determine the number of roots in a given domain by integrating over the domain contour
    \begin{equation} \label{eq-rootnumber}
      N - P = \frac{1}{2 i \pi} \int_{C} \frac{\hat\epsilon'(\omega)}{\hat\epsilon(\omega)} d\omega
    \end{equation}
    where $N$ and $P$ denote the number of roots and poles in the contour $C$.
    Supposing that there are no poles, we either have
    \begin{enumerate}
      \item $N=0$, hence no roots are presents in the domain
      \item $N=1$, exactly one root is present
      \item $N>1$, more than one root are present
    \end{enumerate}

    Starting from a large rectangular domain, if $N>1$, we divide the first domain into four sub-domains, and we repeat recursively the algorithm.
    If $N=1$, we can once again use an integral on the contour to find the root \citep{fortune2001}.
    
    \vspace{1em}
    This algorithm have been implemented in a python package and successfully tested.
    However, it takes a significant amount of time to obtain the solutions, as $\hat\epsilon$ needs to be evaluated a lot of time during the integrations.
    Moreover, we have observed that the dispersion relations \cref{eq-drECDI,eq-MIAW} present only one solution with a positive growth rate.
    This solution, corresponding to the instability, is isolated from the others as observed in \cref{fig-numZbis}.
    Hence, a simpler algorithm, as the Gradient descent, can be used.
    
  \paragraph{Fast root finding algorithm\\}
    A faster root finding algorithm is proposed to solve the relation dispersion by supposing that the most growing wave is the only root over a domain sufficient large.
    In other words, it is not close to other roots.
    Hence, we can use an standard minimization method for non-linear equation.
    As the analytic expression of the Hessian or the gradient are unknown, we use the Nelder-Mead method \citep{mckinnon1998}.
    We also tried the Conjugate gradient method by approximating the gradient using finite differences. 
    However, even if this methods converges in fewer steps, the gradient estimation take a significant amount of time.
    Powell's method \citep{powell1964} has also been implemented, but the Nelder-Mead method present the best performances.
    
    The first guess of the iterative Nelder-Mead method is either 
    \begin{itemize}
      \item the solution obtained for the previous value of $\vect{k}$, 
      \item the solution of the analytic ion acoustic wave dispersion relation (\cref{eq-MIAW}).
    \end{itemize}
    In addition, we can see in \cref{fig-numZbis} that the interesting root is far from the others in the complex plane.
    Hence, a poor initial guess should not affect significantly the converged results, as long as the step size is small enough.
    
% !TEX root=/home/tavant/these/manuscript/src/manuscript.tex

\section{Results}
  \label{sec-DR-results}
  
  
% !TEX root=/home/tavant/these/manuscript/src/manuscript.tex

\section{Radial boundary conditions}
  \label{sec-DR-BC}
  
  In this section, we investigate the interaction between the azimuthal instability and the wall in the radial direction.
  
  \subsection{Impact of the wall on the oscillation}
  \label{subsec-kr}
  
  In the \ac{ECDI} dispersion relation described in \cref{sec-DR-kinetic}, the radial wavenumber is non-zero.
  Hence, in \cref{sec-DR-results}, we have chosen a wavelength that fits between the two walls.
  On the other hand, the oscillation seen in \cref{fig-phi_fluctuation_summary} seems not to present any oscillation in the radial direction.
  \Cref{fig-phi_osci_profile} shows the amplitude of the azimuthal instability on the plasma potential.
  It is defined as
  \begin{equation} \label{eq-stdphi}
    \delta \phi^2 = 2 \sigma_{\phi}^2 = \frac{{2}}{L_{\theta}} \int_0^{L_{\theta}} \lp  \phi - <\phi>_{\theta}  \rp ^2 d\theta,
  \end{equation}
  
  \begin{figure}[hbtp]
    \centering
    \includegraphics[width=\textwidth]{phi_oscillation}
    \caption{(Left) Radial profile of the mean plasma potential average in the azimuthal direction  and in time during the steady-state of the simulation ($t > 3.5 \,\micro\second$) and the average azimuthal instability amplitude. (Right) Radial profile of the  azimuthal instability amplitude, compared with a sinusoidal profile. }
    \label{fig-phi_osci_profile}
  \end{figure}
  
  
  We can see that the amplitude of the oscillation in the radial direction does not follow a sinusoidal profile.
  This observation contradicts the hypothesis made previously on the value of $k_r$.
  In order to better apprehend the radial profile of the instability amplitude, we show in \cref{fig-ratio} the ratio between $\delta \phi$ the fluctuation amplitude (showed on the right in \cref{fig-phi_osci_profile}) and $<\phi>_{\theta}$  the mean plasma potential (on the left in \cref{fig-phi_osci_profile}).
  
  \begin{figure}[hbtp]
    \centering
    \includegraphics[width=\defaultwidth]{phi_oscillation_ratio}
    \caption{Radial profile of the ratio between $\delta \phi$ the fluctuation amplitude (showed on the right in \cref{fig-phi_osci_profile}) and $<\phi>_{\theta}$  the mean plasma potential (on the left in \cref{fig-phi_osci_profile}).}
    \label{fig-ratio}
  \end{figure}
  
  \Cref{fig-ratio} shows that the ration $\delta \phi / <\phi>$ is almost constant in the plasma, except close to the wall, where the plasma potential decreases much faster than the oscillations.
  The same behavior can bee observed in the profiles of the ion density, that can bee seen in \Cref{fig-ion_oscilation}.
  Actually, the behavior even more pronounced for the ion density, as they are not significantly affected by the wall.
  The ratio between the fluctuations and the mean value presents a radial profile almost constant.
  
  \begin{figure}[hbtp]
    \centering
    \includegraphics[width=\textwidth]{Ion_oscilations.pdf}
    \caption{Radial profile of (left) the mean ion density profile and $\delta n_i$ the fluctuation amplitude and (right) the ratio between $\delta n_i$ and $n_i$.}
    \label{fig-ion_oscilation}
  \end{figure}
  
  \vspace{1em}
  This may suggest that the sheaths screen significantly the wall from the oscillations, that are not affected by it.
  A screening of the wall have been observed in \citet{janhunen2018}, they observed radial structures that are not observed here.
  The difference between  \citet{janhunen2018} and the results presented here may be due to the difference in the radial length ($L_R =1 \,\centi\meter$ here against $L_R = 5.38\,\centi\meter$).
  
  As a conclusion, the interaction between the instability and the boundaries is not clearly understood.
  In addition, if there is no radial structure observed in our simulations, it means that the oscillation is purely azimuthally, or at least $\lde k_r <<1$.
  In this case, the cyclotron resonances should non-longer disappear \citep{ducrocq2006}, except due to non-linear demagnetisation, as discussed before \citep{boeuf2018,taccogna2019}.
  However, non-linear dispersion relations are out of the scope of the present work.
  
  
  \subsection{Impact of the electrostatic condition}
  \label{subsec-BC}
  \inlinenote{maybe this should be in Ch2: \\ Anne: yes }
  
  Let now observe the impact of the dielectric electrostatic boundary condition on the oscillation.
  We have see in \Cref{ch-2} that the dielectric boundary did not affect the simulation macroscopic results.
  \Cref{fig-closswallosci} shows the radial evolution in the first few cells of the amplitude of the oscillation of the azimuthal electric field on the left, and the ion density on the right, with grounded (metallic) wall and dielectric wall.
  
  \begin{figure}[hbtp]
    \centering
    \includegraphics[width=\textwidth]{Ex_closewall.pdf}
    \caption{Radial evolution in the first cells of the amplitude of the oscillation of (left) the azimuthal electric field and (right) the ion density, with grounded (metallic) wall and dielectric wall.}
    \label{fig-closswallosci}
  \end{figure}
  
  We can see in \cref{fig-closswallosci} that the boundary condition do not affects the ion oscillations.
  This is consistent with the observation in \cref{subsec-kr} that the ion fluctuation was not affect by the wall.
  On the other hand, the azimuthal electric field has to go to zero when the wall is grounded.
  In contrast, the dielectric boundary condition, as modeled, allows a non-zero azimuthal electric field at the wall limit.
  However, the difference quickly disappears in the plasma.
  Hence, the electrostatic boundary condition induces only minor differences on the plasma discharge.
  
  
% !TEX root=/home/tavant/these/manuscript/src/manuscript.tex

\section{Conclusion}
  Azimuthal instabilities have been observed in the \ac{PIC} simulations presented here, as well as by several groups of the community \citep{hara2019a,janhunen2018,taccogna2019}.
  As their nature remain unclear, we investigated in this chapter the \ac{PIC} simulation results in order to obtain more insights on their nature and their behavior.
  The theoretical dispersion relations for the \ac{ECDI} and the \ac{IAW} have been used in both their simplified and general forms.
  The general form of the dispersion relation uses the particle velocity distribution function directly measured in the \ac{PIC} simulations.
  The solver used is presented in \cref{sec-DR-solver}.
  
  At the beginning, the resonances typical of the \ac{ECDI} are observed.
  After about $0.7\,\micro\second$, they disappear, evolving towards the \ac{IAW} dispersion relation.
  However, we show that the evolution from the \ac{ECDI} to the \ac{IAW}  is not due to the larger radial wavenumber, as we observe no radial pattern in the simulation.
  Instead, it might be due to non-linear resonance broadening and electron demagnetization.
  
  \vspace{1ex}
  We also observed low frequency modulations of the amplitude of the instability.
  This oscillation is believed to be driven by the ion dynamic of the the ion-wave trapping.
  Indeed, the oscillation period is of the order of $\tau=1.5\,\micro\second$, which is four times the ion bouncing period.
  
  The oscillation of the growth rate is also observed with the dispersion relation when the ion temperature is used and that they are supposed Maxwellian.
  However, when the ion velocity distribution function measured in the \ac{PIC}  simulation is used, the growth rate stays almost constant, with small oscillations.
  The growth rate in the simulation is estimated with the wave equation.
  The measured growth rate is in better agreement with the dispersion relation assuming Maxwellian distribution function, compared to the one using the VDF measured int he PIC simulation.
  
  The origin of the discrepancy is unclear, as there are several possible reason.
  Firstly, to obtain the dispersion relations used here, we assumed small oscillations.
  However, the quasi-steady state is governed by non-linear phenomena, which probably have an affect the dispersion relation.
  Secondly, the estimation of the growth rate in the simulation uses a crude estimation of impact of the axial direction and the convection model on the wave.
  Lastly, the general dispersion relation solver use here has not been validated with distribution functions of complex shape, as the one observed in \cref{fig-ivdfs_pic_time}.
  A proper cross-validation with other solvers using complex distribution function is needed to confirm the result obtained.  
  Another approach for the dispersion relation solver is to fit the VDF using a sum of analytic functions.
  Usually, one or few simple distribution functions are used, such as done by \citet{ronnmark1982}.
  A general algorithm finding the best fit in a large ensemble of analytic functions could be used to analyze more complex distribution function as the one observed here in the PIC simulation.

  \vspace{1ex}
  The results observed here are globally similar with the other simulation results of the community discussed in \cref{subsec-indroECDI}.
  However, some differences remain, in particular during the non-linear stage.
  In order to conclude on the origins of the discrepancies, a precise comparison between the simulation codes and the parameters used will be conducted.
  A Benchmark is currently in construction, to compare some of these codes on the radial-azimuthal geometry similarly to the work done on the axial-azimuthal domain \citep{charoy2019}.
  
  

