% !TEX root=/home/tavant/these/manuscript/src/manuscript.tex

\section{Conclusion}
  Azimuthal instabilities have been observed in the \ac{PIC} simulations.
  As their nature remain unclear, we investigated the simulation results in order to obtain more insights.
  
  The theoretical dispersion relations for the \ac{ECDI} and the \ac{IAW} have been used in both their simplified and general forms.
  The general form of the dispersion relation uses the particle velocity distribution function directly measured in the \ac{PIC} simulations.
  The solver use is presented in \cref{sec-DR-solver}.
  
  At the beginning, the resonances typical of the \ac{ECDI} are observed.
  The, they disappear, living place to the \ac{IAW}.
  However, we show that the evolution from the \ac{ECDI} to the \ac{IAW}  is not due to the larger radial wavenumber.
  Instead, it due to non-linear resonance broadening and electron demagnetisation.
  
  \vspace{1ex}
  We also observed low frequency modulation of the amplitude of the instability.
  This oscillation is impacted by the ion latency of being trapped by the wave.
  The oscillation period is of the order of $\tau=1.5\,\micro\second$, which is slower than the bouncing frequency or the axial transit time.
  
  The oscillation at the growth rate is also observed with the dispersion relation when the ion temperature is used, but that they are supposed Maxwellian.
  When the ion velocity distribution function measured in the \ac{PIC}  simulation is used, the growth rate stays close to zero.
  However, This could be due to errors in the general relation dispersion solver.
  A proper cross-validation with other solver using discrete distribution function is needed.
  
  Another approach, not yet used here, would be to fit the distribution function using a sum of analytic functions.
  Usually, one or few simple distribution functions are used.
  A general algorithm finding the best fit in large ensemble of analytic function could be used to analyse more complex distribution function a the one observed in the PIC simulation.
  
  \vspace{1ex}
  Lastly, we observed that the dielectric boundary condition affects the azimuthal electric field close to the wall, but the sheath screen quickly the walls from the plasma.
  Consequently, as in \cref{ch-2}, we observed no major impact of the dielectric boundary condition on the instability in our simulation.
  
  The results observed here are global similar with the other simulation results of the community.
  However, some significant difference remains, in particular during the non-linear saturation stage.
  In order to conclude on the origin of the discrepancy, a precise comparison between the simulation codes and the parameters used should be done.
  A Benchmark is currently in the process of comparing some of these code on the axial and azimuthal geometry.
  Once it is finished, the effort will be made on the radial and azimuthal case.
  
  
