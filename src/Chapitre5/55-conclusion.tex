% !TEX root=/home/tavant/these/manuscript/src/manuscript.tex

\section{Conclusion}
  Azimuthal instabilities have been observed in the \ac{PIC} simulations presented here, as well as by several groups of the community \citep{hara2019a,janhunen2018,taccogna2019}.
  As their nature remains unclear, we investigated in this chapter the \ac{PIC} simulation results in order to obtain more insights on their nature and their behavior.
  The theoretical dispersion relations for the \ac{ECDI} and the \ac{IAW} have been used in both their simplified and general forms.
  The general form of the dispersion relation uses the particle velocity distribution function directly measured in the \ac{PIC} simulations.
  The solver develop and used used is presented in \cref{sec-DR-solver}.
  
  At the beginning, the resonances typical of the \ac{ECDI} are observed.
  After about $0.7\,\micro\second$, they disappear, evolving towards the \ac{IAW} dispersion relation.
  However, we show that the evolution from the \ac{ECDI} to the \ac{IAW}  is not due to the larger radial wavenumber, as we observe no radial pattern in the simulation.
  Instead, it might be due to non-linear resonance broadening and electron demagnetization.
  
  \vspace{1ex}
  We also observed low frequency modulations of the amplitude of the instability.
  This oscillation is believed to be driven by the ion dynamic of the ion-wave trapping.
  Indeed, the oscillation period is of the order of $\tau=1.5\,\micro\second$, which is four times the ion bouncing period.
  
  The oscillation of the growth rate is also observed with the dispersion relation when the ion temperature is used and that ions are supposed Maxwellian.
  However, when the ion velocity distribution function measured in the \ac{PIC}  simulation is used, the growth rate stays almost constant, with small oscillations.
  The growth rate in the simulation is estimated with the wave equation.
  The measured growth rate is in better agreement with the dispersion relation assuming Maxwellian distribution function, compared to the one using the VDF measured int he PIC simulation.
  
  The origin of the discrepancy is unclear, as there are several possible reason.
  Firstly, to obtain the dispersion relations used here, we assumed small oscillations.
  However, the quasi-steady state is governed by non-linear phenomena, which probably have an affect the dispersion relation.
  Secondly, the estimation of the growth rate in the simulation uses a crude estimation of impact of the axial direction and the convection model on the wave.
  Lastly, the general dispersion relation solver use here has not been validated with distribution functions of complex shape, as the one observed in \cref{fig-ivdfs_pic_time}.
  A proper cross-validation with other solvers using complex distribution function is needed to confirm the result obtained.  
  Another approach for the dispersion relation solver is to fit the VDF using a sum of analytic functions.
  Usually, one or few simple distribution functions are used, such as done by \citet{ronnmark1982}.
  A general algorithm finding the best fit in a large ensemble of analytic functions could be used to analyze more complex distribution function as the one observed here in the PIC simulation.

  \vspace{1ex}
  The results observed here are globally similar with the other simulation results of the community discussed in \cref{subsec-indroECDI}.
  However, some differences remain, in particular during the non-linear stage.
  In order to conclude on the origins of the discrepancies, a precise comparison between the simulation codes and the parameters used will be conducted.
  A Benchmark is currently in construction, to compare some of these codes on the radial-azimuthal geometry similarly to the work done on the axial-azimuthal domain \citep{charoy2019}.
  
  
