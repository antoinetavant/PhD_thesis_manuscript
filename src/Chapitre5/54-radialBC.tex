% !TEX root=/home/tavant/these/manuscript/src/manuscript.tex

\section{ Discussion on the radial wavenumber}
  \label{sec-DR-BC}
    
  In the \ac{ECDI} dispersion relation introduced in \cref{sec-DR-kinetic}, the radial wavenumber must be non-zero.
  Hence, in \cref{sec-DR-results}, we have chosen a wavelength that fits between the two walls.
  On the other hand, the oscillation seen in \cref{fig-phi_fluctuation_summary} does not seem to present any oscillation in the radial direction.
  In this section, we investigate the interaction between the azimuthal instability and the wall in the radial direction.
  
  \subsection{Radial profile of the oscillation} \label{subsec-radial_prof}

  In this section, we analyze the radial profile of the azimuthal instability described in the previous sections.
  \Cref{fig-phi_osci_profile} shows the amplitude of the azimuthal instability on the plasma potential.
  It is defined as
  \begin{equation} \label{eq-stdphi}
    \delta \phi^2 = 2 \sigma_{\phi}^2 = \frac{{2}}{L_{\theta}} \int_0^{L_{\theta}} \lp  \phi - <\phi>_{\theta}  \rp ^2 d\theta,
  \end{equation}
  
  \begin{figure}[!hbt]
    \centering
    \includegraphics[width=\textwidth]{phi_oscillation_bis}
    \caption{(Left) Radial profile of the mean plasma potential averaged in the azimuthal direction  and in time during the steady-state of the simulation ($t > 3.5 \,\micro\second$) and the average azimuthal instability amplitude. (Right) radial profile of the relative importance of the amplitude of the fluctuation to the mean plasma potential $<\phi>$. }
    \label{fig-phi_osci_profile}
  \end{figure}
  
  \cref{fig-phi_osci_profile} shows on the left the mean plasma potential $<\phi>$ averaged in time and in the azimuthal direction , and the amplitude of the fluctuation at each radial position.
  We see for $<\phi>$ the usual profile with the sheaths close to the wall characterized by a steep potential drop, and the presheaths in the plasma bulk with a more gentle variation.
  However, we see no radial pattern in the plasma potential nor the amplitude of the azimuthal plasma potential oscillation.
  
  On the right panel of \cref{fig-phi_osci_profile}, we see the relative instability amplitude with respect to the mean plasma potential $\delta\phi / <\phi>$, in percent.
  We see that the amplitude of the oscillation follows the mean value of the plasma potential.
  Indeed, except close to the wall where $\phi$ reaches zero, the amplitude of the oscillation is approximately 8\% of $\phi$ in all of the simulation domain.
  The oscillation to mean value ratio is compared to the analytic $\delta \phi \propto \sin( \pi \frac{r}{L_R})$ in \cref{fig-phi_osci_profile}.
  The sin profile corresponds to a radial wavelength twice as large large as the radial width.
  We can see that the sin profile decreases much faster toward the wall, compared to the simulation result.
  
  

  The same constant ratio between the amplitude of the oscillation and the mean value is observed in the ion density, that can be seen in \Cref{fig-ion_oscilation}.
  As a mater of fact, the phenomena is even more pronounced on the ion density, as the ratio $\delta n_i / n_i$ presents a radial profile almost constant everywhere, even at the wall.
  It can be explains by the fact the ions reaches the wall with a supersonic speed, so that the wall does not directly affect the ions.
  
  
  \begin{figure}[!hbt]
    \centering
    \includegraphics[width=\textwidth]{Ion_oscilations.pdf}
    \caption{Radial profile of (left) the mean ion density profile and $\delta n_i$ the fluctuation amplitude and (right) the ratio between $\delta n_i$ and $n_i$.}
    \label{fig-ion_oscilation}
  \end{figure}
  
  \vspace{1em}
  The fact that the amplitude of the oscillation is proportional to the mean azimuthal value may suggest that the sheaths significantly  screen  the oscillation from the walls .
  Therefore, the instability is not affected by the walls.
  A screening of the wall have been observed in \citet{janhunen2018}, where the authors observed radial structures of wavelength larger than the radial length.
  However, here we observe no radial structure at all.
  The difference between  \citet{janhunen2018} and the results presented here may be due to the difference in the radial length ($L_R =1 \,\centi\meter$ is used here, against $L_R = 5.38\,\centi\meter$ in \citet{janhunen2018} ).
  
  \subsection{Impact of the radial wavenumber on the \acs{DR}}
   \label{subsec-kr}

  To highlight the impact of the radial wavenumber on the \ac{ECDI} \ac{DR}, \cref{fig-kreffect} shows for three values of $k_r \lde$ the evolution of the frequency $\omega$, and  the growth rate $\gamma$ as a function of the azimuthal wavenumber $k_{\theta}$.
  We see that when $k_r \lde$ increases from 0.02 to 0.1, the cyclotron resonances decrease and broaden until they disappear.
  \begin{figure}[!hbt]
    \centering
    \includegraphics[width=\defaultwidth]{ECDI_ktheta_impact.pdf}
    \caption{Evolution as a function of the azimuthal wavenumber $k_{\theta}$ of (solid line) the frequency $\omega$, and (dashed line) the growth rate $\gamma$ for three values of the normalized radial wavenumber $k_r \lde$ for the \acs{ECDI} with Maxwellian electrons and cold ions. }
    \label{fig-kreffect}
  \end{figure}
  
  This reduction of the resonances explains the observations of \cref{fig-DRECDI}, where we saw similar reduction of the cyclotron resonances, due to the increase of $\lde$ from $\lde=\sn{4.3}{-5}\,\meter$ at $t=0$ to $\lde=\sn{7.0}{-5}\,\meter$ at saturation.
  However, if the radial wavenumber $k_r$ goes to zero, in agreement with the observations of \cref{fig-ion_oscilation}, then the resonances would not smooth out for this reason.
  Instead, the resonances would stay.
  
  \vspace{1em}
  As a conclusion, the interaction between the instability and the boundaries is not clearly understood.
  The sheaths seem to  screen the waves from the radial boundaries.
  In \Cref{subsec-BC}, we will discuss again the impact of the radial boundary condition (metallic versus dielectric electrode) on the oscillation.

  As there is no radial structure observed in the simulation showed here, it means that the oscillation is purely azimuthal, or at least $\lde k_r <<1$.
  In this case, the cyclotron resonances should non-longer disappear \citep{ducrocq2006}, except due to non-linear demagnetization of the electrons, as discussed before \citep{boeuf2018,taccogna2019}.
  The non-linear dispersion relations are out of the scope of the present work, but should be developed in order to better understand the evolution on the \ac{ECDI} in the \ac{HET}.
  
  To understand the divergent observations of \citet{hara2019a,janhunen2018}, and \citet{taccogna2019}, a comparison of the simulation results using similar parameter and models should be undertaken.
  A comparison is currently in progress between the different working groups on another simulation case \citep{charoy2019}. 
  Once this first Benchmark is conducted, the radial-azimuthal geometry should be studied.
  
  
  
  