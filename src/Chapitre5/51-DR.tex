% !TEX root=/home/tavant/these/manuscript/src/manuscript.tex

\section{Dispersion relation of the instabilities}
  \label{sec-DR-kinetic}
  
  
  The dispersion relations are obtained by coupling the particle dynamics with the electric fields.
  In the case of the kinetic electrostatic dispersion relation, we couple the Vlasov equation with the Poisson equation.
  
  In our \ac{2D} geometry, we can neglect all of the gradients.
  A fixed axial electric field $\vect{E} = E_0 \vect{e_z}$ and a fixed radial magnetic field $\vect{B} = B_0 \vect{e_r}$ are imposed.
  The electrons are drifting in the azimuthal direction due to the $E\times B$ drift, hence
  \[ \vect{u_{e}} = \frac{\vect{E} \times \vect{B}}{\norm{\vect{B}}^2} = \frac{E_0}{B_0}  \vect{e_{\theta}}.    \]
  The ions are accelerated in the axial direction, $ \vect{u_i} = u_i  \vect{e_z}$.
  
  The perturbations are waves of the form \[ \exp \lp i \vect{k} \cdot \vect{x} - i \omega t  \rp, \]
  with $\vect{x}$ the position vector, the oscillation wave vector $\vect{k} = k_r \vect{e_r} + k_{\theta} \vect{e_{\theta}}$ is real but its frequency $\omega = \omega_r + i \gamma$ can be complex, where $\gamma$ is the growth rate of the oscillations. 
  
  \vspace{1em}
  The dispersion relation has been studied by \citet{ducrocq2006} in the case of cold ions and Maxwellian electrons in a \ac{2D} geometry.
  A numerical algorithm has been proposed by \citet{cavalier2013} and compared to experimental measurements.
  In \citet{lafleur2016}, the authors added the ion drift velocity to the dispersion relation.
  
  The \ac{ECDI} dispersion relation obtained presents resonances at the cyclotron frequencies, which broadens when $k_r$ in increased.
  The limit when $k_r$ tends to large values is similar to an \ac{IAW}.
  This limit is usually used, even if recently, \ac{2D} \ac{PIC} simulations with a larger domain observed radial structures in the oscillations \citep{janhunen2018,hara2019a}.
  
  \citet{lafleur2018} removed the Maxwellian hypothesis by using directly the distribution functions measured in the \ac{PIC} simulations in the \ac{IAW} dispersion relation.
  The authors showed that the frequency of the oscillation is almost unperturbed, but the growth rate is significantly reduced, even when the ions were still supposed Maxwellian \citep[Fig. 8]{lafleur2018}.
  
  Here, we propose to continue the investigation by solving the dispersion relation numerically with the electron and ion distribution function for both the \ac{IAW} and the \ac{ECDI}.
  
  
  \subsection{General dispersion relation}
    \label{sec-geneDR}
    
    We follow the development presented in \citet{ducrocq2006,cavalier2013,lafleur2016}.
    The plasma dielectric function is defined as
    \begin{equation} \label{eq-de}
      \hat\epsilon(\vect{k},\omega) = 1 - \sum_s \chi_s(\vect{k},\omega)
    \end{equation}
    
    where $\chi_s(\vect{k}, \omega)$ is the susceptibility of the species $s$.
    It is obtained by coupling the Poisson equation with the particles description.
    The dispersion relation is obtained by setting $  \hat\epsilon(\vect{k},\omega) =0$ ans solving for $\vect{k},\omega$.
    
    
    For the unmagnetized ions, supposing a Maxwellian distribution, the susceptibility is
    \begin{equation} \label{eq-susceptibility}
      \chi_i(\vect{k},\omega) = \frac{\opi^2}{k^2 v^2_{th, i}} Z'\lp \frac{\omega - \vect{k} \cdot \vect{u_{i}}}{k v_{th, i}}  \rp
    \end{equation}
    where $\opi$ the ion plasma pulsation, $k=\norm{\vect{k}}$ and $\vect{u_i}$ is the mean velocity of the ions.
    The function $Z'$ is the derivative of the Fried and Conte function \citep{fried1961}
    \begin{equation} \label{eq-friedandConte}
      Z(\eta) = \frac{1}{\sqrt{\pi}} \int_{-\infty}^{\infty} \frac{\exp{(-t^2)}}{t - \eta} dt.
    \end{equation}
    We use here the Fried and Conte function because of the Maxwellian hypothesis.
    \citet{xie2013} proposes a numerical algorithm to calculate the susceptibility for a general distribution function
    \begin{equation} \label{eq-general}
      Z(\eta, f) = \int_{-\infty}^{\infty} \frac{f(t)}{t - \eta} dt,
    \end{equation}
    with $f$ the velocity distribution function to consider.
    For sake of brevity, the generalized dispersion function $Z(\eta,f)$ is noted $Z(\eta)$, and the Fried and Conte function is noted $Z_M(\eta)$.
    The derivative of $Z$ is
    \begin{equation} \label{eq-derivatives}
      Z'(\eta) = \int_{-\infty}^{\infty} \frac{\partial f(t) / \partial t}{t - \eta} dt,
    \end{equation}
    
    \vspace{1em}
    A general expression for the plasma dielectric function can be obtained for magnetized electrons by making use of the method of characteristics and is given by
    \begin{equation} \label{eq-drECDI}
      \begin{split}
      \hat\epsilon(\vect{k},\omega) =& 1 - \\
       &\frac{\opi^2}{k^2 v^2_{th, i}} Z'\lp \frac{\omega - \vect{k} \cdot \vect{u_{i}}}{k v_{th, i}}  \rp + \\
       & \frac{1}{k^2 \lde^2} \lb 1 + \lp  \frac{\omega - \vect{k} \cdot \vect{u_{e}}}{k v_{th, e}} \rp \sum_{n=-\infty}^{\infty} e^{- \beta} I_n(\beta) Z\lp  \frac{\omega - \vect{k} \cdot \vect{u_{e} - n \oce}}{k_{r} v_{th, e}} \rp  \rb,
    \end{split}
    \end{equation}
    where $I_n$ are the modified Bessel functions of the first kind, and 
    \begin{equation} \label{eq-beta}
      \beta = \frac{(k_{\theta}^2 + k_z^2) b^2_{th, e}}{ \oce^2}
    \end{equation}
    
    \Cref{eq-drECDI} is the dispersion relation for drifting magnetized electrons and unmagnetized ions.
    It will be used to study the \acf{ECDI}.
    


  \subsection{Modified Ion Acoustic Wave}
    \label{sucsec-IAW}
    
    The dispersion relation of \cref{eq-drECDI} presents sharp resonances due to cyclotron resonances.
    However, kinetic simulations in \citet{janhunen2018,taccogna2019} have shown that after some time (around  $t=0.5\,\micro\second$ and $t=3\,\micro\second$ respectively) the resonances are no longer present, with a progressive decrease of the higher harmonics and the first harmonics becoming the most prominent.
    Without the resonances, the dispersion relation evolves to the nonmagnetic ion-acoustic instability
    \begin{equation} \label{eq-drIAWgene}
      \begin{split}
      \hat\epsilon(\vect{k},\omega) =& 1 - \\
       &\frac{\opi^2}{k^2 v^2_{th, i}} Z'\lp \frac{\omega - \vect{k} \cdot \vect{u_{i}}}{k v_{th, i}}  \rp + \\
       & \frac{1}{k^2 \lde^2}  Z'\lp  \frac{\omega - \vect{k} \cdot \vect{u_{e}}}{k v_{th, e}} \rp ,
    \end{split}
    \end{equation}

    \citet{lafleur2016} and \citet{janhunen2018} show that after some assumptions -- mostly a drifting Maxwellian distribution, cold ions, and a small electron drift velocity compared to thermal speed -- \cref{eq-drIAWgene} can be solved to obtain
    \begin{equation} \label{eq-MIAW}
      \omega = \omega_r + i \gamma = \vect{k} \cdot \vect{u_{i}} \pm \frac{k c_s}{\sqrt{1 + k^2 \lde^2}} \pm i \sqrt{\frac{\pi m_e}{8 m_i}} \frac{\vect{k} \cdot \vect{u_{e}}}{( 1 + k^2 \lde^2)^{3/2}}.
    \end{equation}
    The above equation represents the analytic modified ion-acoustic dispersion relation.
    The wave-number given the maximum of the growing rate is \citep{lafleur2016}
    \begin{equation} \label{eq-mostk}
      k_{max} = \frac{1}{\sqrt{2} \lde}
    \end{equation}
    which gives, subtituted in \cref{eq-MIAW}
    \begin{equation} \label{eq-mostw}
      \omega_{max} =  \vect{k} \cdot \vect{u_{i}} \pm \frac{\opi}{\sqrt{3}} + i \sqrt{ \frac{\pi m_e}{54 m_i}}\frac{u_e}{\lde}
    \end{equation}
    We can see that the growth rate is proportional to the electron drift velocity, and inversely proportional to the Debye length.
    
    However, one should note that there is no consensus on the transition to an \ac{IAW}.
    It can be is attributed to non-linear resonance broadening, because the electron orbit is distorted, leading to the loss of the phase relation between the electron and the wave \citep{taccogna2019}.
    The electron demagnetization leads to a dispersion relation where the ions and the electrons have the same type of contribution.
    But recently, \citet{janhunen2018a} stated that the demagnetization condition due to nonlinear resonance broadening is not fulfilled.
    \citet{lafleur2017a} uses the fact that in \ac{2D} simulations, the finite minimum value for the radial wave vector is responsible for the ion acoustic dispersion relation.
    
    