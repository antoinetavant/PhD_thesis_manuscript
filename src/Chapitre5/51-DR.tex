% !TEX root=/home/tavant/these/manuscript/src/manuscript.tex

\section{Dispersion relation of the instabilities}
  \label{sec-DR-kinetic}
  
  
  The dispersion relations are obtained by coupling the particle dynamics with the electric fields.
  In the case of the kinetic electrostatic dispersion relation, we couple the Vlasov equation with the Poisson equation.
  
  In our \ac{2D} geometrie, we can neglect all of the gradients.
  Are imposed a fixed axial electric field $\vect{E} = E_0 \vect{e_z}$ and a fixed radial magnetic field $\vect{B} = B_0 \vect{e_r}$
  The electrons are drifting in the azimuthal direction due to the $E\times B$ drift, hence
  \[ \vect{u_{e}} = \frac{\vect{E} \times \vect{B}}{\norm{\vect{B}}^2} = \frac{E_0}{B_0}  \vect{e_{\theta}}.    \]
  The ions are accelerated in the axial direction, $ \vect{u_i} = u_i  \vect{e_z}$.
  
  The perturbations are waves of the form \[ \exp \lp \vect{k} \cdot \vect{x} - \omega t  \rp, \]
  with $\vect{x}$ the position vector, the oscillation wave vector $\vect{k}$ is real but its frequency $\omega = \omega_r + i \gamma$ can be complex, where $\gamma$ is the growth rate of the oscillations. 
  
  \vspace{1em}
  The dispersion relation has been studied by \citet{ducrocq2006} in the case of cold ions and Maxwellian electrons in a \ac{2D} geometry.
  A numerical algorithm has been proposed by \citet{cavalier2013} and compared to experimental measurements.
  In \citet{lafleur2016}, the authors added the ion drift velocity to the dispersion relation.
  
  The \ac{ECDI} dispersion relation obtained presents resonances at the cyclotron frequencies, which broadens when $k_r$ in increased.
  The limit when $k_r$ tends to large values is similar to an \ac{IAW}.
  This limit is usually used.
  
  Recently, \ac{2D} \ac{PIC} simulation on a large domain observed radial structures in the oscillations \citep{janhunen2018,hara2019a}.
  
  \citet{lafleur2018} got read of the Maxwellian electron hypothesis by using directly the electron distribution function measured in the \ac{PIC} simulations in the \ac{IAW} dispersion relation.
  He showed that the frequency of the oscillation is almost unperturbed, but the growth rate is significantly reduced, even when the ions were still supposed Maxwellian.
  
  Here, we propose to continue the investigation by solving the dispersion relation numerically with the electron and ion distribution function for both the \ac{IAW} and the \ac{ECDI}.
  
  