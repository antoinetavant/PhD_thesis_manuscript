% !TEX root=/home/tavant/these/manuscript/src/manuscript.tex

\section{Conclusion of the parametric study}
  \label{sec-conclusion_ch2}
  
  Using the \ac{PIC} simulation code introduced in \cref{ch-1}, we studied the effects of the dielectric walls on the discharge, and more precisely the effects on the electron axial mobility.
  To begin with, a \emph{base} case with metallic walls was defined and studied.
  The metallic walls correspond to grounded and non-emissive walls.
  For this reference case, we observed that the convection model used allows us to obtain a quasi steady-state.
  We observed an enhanced electron transport transverse to the magnetic field lines, because of the azimuthal instability.
  Both effects of the dielectric -- the electron induced electron emission and the electrostatic boundary condition -- were investigated.
  First, we  only modeled the dielectric boundary condition. Then, we studied only the electron emission. Afterwards, the two phenomena have been studied together.
  
  \subsubsection*{Electrostatic boundary condition}
  
  The electrostatic boundary condition is modeled by including in the domain of simulation the thickness of the wall ($L_{\rm Diel} = 3 \milli\meter$).
  Surface charges accumulate at the interface between the plasma and the wall.
  We observed that the modified boundary condition did not modify significantly the discharge and the axial cross-field electron mobility.
  Moreover, we saw that the boundary condition used results in a radial electric field $E_r$ of the same order of magnitude than the Neumann boundary condition of \cref{eq-neuman} but the spatio-temporal evolution is not identical.
  
  Indeed, when the secondary electron emission is modeled, the surface charges oscillates significantly compared to the radial electric field during the \ac{SCL} regime.
  As the dielectric model used here does not increase significantly the computational time, we recommend to use it instead of the Neumann boundary condition, that do not reproduce the same plasma-wall interaction.
  
  
  \subsubsection*{Electron induced electron emission}
  
  The electron  emission from the wall due to the impact of primary electrons reaching the wall is modeled using the model described in \cref{sec-seemodel}.
  The value of the crossover energy $\crover$ is varied from a large value (low emissivity) to small values (high emissivity).
  We observed in the simulations that when the electron emission rate increases, the mean electron temperature decreases.
  This decreases the amplitude of the \ac{ECDI} at saturation, hence decreases the electron mobility in the plasma (see \cref{fig-radial-data}).
  However, electron emission induces \ac{NWC}, which almost doubles the electron mobility close to the wall when $\crover$ varies from $200\volt$ to $30\volt$.
  Consequently, the overall electron cross-field mobility is almost constant in our simulation.
  
  We observed in our \ac{PIC} simulations three different regimes depending on the values of $\crover$.
  For high values of $\crover$, the plasma stabilises with an emission rate $\ratepic < \ratecr$.
  When $\crover$ is small, we observe a stable configuration with $\ratepic \sim \ratecr$.
  Under these conditions, the sheath is space-charge limited.
  The transition between the two regimes is not stable, but instead passes by a bi-stable regime.
  In this third regime, the sheath oscillates between the two stable regimes.
  

  \subsubsection*{Comparison with classical sheath model}
  
  The simulation results have been compared to the classical sheath model of \citet{hobbs1967}.
  We observed a significant discrepancy between the \ac{PIC} simulations and the sheath model that comes from a fluid approach.
  In particular, the potential drop and the electron emission rate are both overestimated.
  These overestimations can lead to erroneous conclusion and prediction when using fluid models.
  Hence, a better understanding of the plasma-wall transition via the sheath is needed.
  
  The sheath model currently used is based mainly on two hypothesis
  \begin{itemize}
    \item Maxwellian electron distribution function
    \item Isothermal electrons in the sheath
  \end{itemize}
  
  Both hypotheses will be questioned in the next chapter.
  
