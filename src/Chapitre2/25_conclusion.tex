% !TEX root=/home/tavant/these/manuscript/src/manuscript.tex

\section{Conclusion}
  \label{sec-conclusion_ch2}
  
  Using the \ac{PIC} simulation code introduced in \vref{ch-1}, we studied the effects of the dielectric walls on the discharge, and more precisely the effects on the electron cross-field mobility.
  
  To begin with, a \emph{canonical} case, with metallic walls was defined and studied.
  The metallic walls correspond to grounded and non-emissive walls.
  For this reference case, we observed that the convection model used allow us to obtain a steady state.
  We observed an enhanced electron transport transverse to the magnetic field lines, because of the instability.
  
  Both effects of the dielectric -- the electron induced electron emission and the electrostatic boundary condition -- were investigated.
  First, we  only modeled the dielectric boundary condition. Then, we studied only the electron emission. Afterwards, the two phenomena have been studied together.
  
  \subsubsection*{Electrostatic boundary condition}
  
  The electrostatic boundary condition were modeled by including in the domain of simulation the thickness of the wall ($L_{\rm wall} = 3 \centi\meter$).
  Surface charges accumulate at the interface between the plasma and the wall.
  We observed that the modified boundary condition did not modify significantly the discharge and the axial electron mobility.
  We saw that the boundary condition used results in a radial electric field $E_r$ of the same order of magnitude than the Neumann boundary condition of \vref{eq-neuman}.
  However, the spatio-temporal evolution are not similar.
  
  More importantly, when the electron emission is modeled as when, the observed behaviour differs significantly from the Neumann condition.
  This leads to differences in the electron emission rate.
  As the dielectric model used here do not increase significantly the computational time, we recommend to use it instead of the Neumann boundary condition, that do not reproduce the same plasma-wall interaction.
  
  
  \subsubsection*{Electron induce electron emission}
  
  Electron induced electron emission from the wall is modeled using a linear model \vref{sec-seemodel}.
  The crossover energy $\crover$ is varied from a large value (low emissivity material) to small values (high emissivity).
  We observed in the simulations that when the electron emission rate increases, the mean electron temperature decreases.
  This decreases the amplitude of the \ac{ECDI} at saturation, hence decreases the electron modbility in the plasma (see \vref{fig-radial-data}).
  However, electron emission induces \ac{NWC}, which almost double the electron mobility close to the wall when $\crover$ passes from $200\volt$ to $30\volt$.
  Consequently, the overall electron crossfield mobility is almost constant in our simulation.
  
  We observed in our simulation three different regimes depending of the values of $\crover$.
  For high values of $\crover$, the plasma stabilised with an emission rate $\ratepic < \ratecr$.
  When $\crover$ is small, we observe a stable configuration with $\ratepic \sim \ratecr$.
  Under these conditions, the sheath in space-charge limited.
  The transition between the two regimes is not stable, but instead passes by a bifurcation regime.
  During this third regime, the sheath jumps between the two stable regimes.
  Indeed, the transition passes through an unstable phase, as we can see it on \vref{fig-dphivsTe}.
  

  \subsubsection*{Inconsistent sheath model }
  
  The simulation results have been compared to the sheath model of \citet{hobbs1967}.
  We observed a significant discrepancy between the \ac{PIC} simulations and the sheath model that comes from a fluid approach.
  In particular, the potential drop and the electron emission rate are both overestimated.
  These overestimation can lead to erroneous conclusion and prediction when using fluid models.
  Hence, a better understanding of the plasma-wall transition via the sheath is needed.
  
  The sheath model currently used reposes mainly on two hypothesis
  \begin{itemize}
    \item Maxwellian electrons,
    \item Isothermal evolution of the electrons in the sheath
  \end{itemize}
  
  Theses hypothesis will be confronted in the next chapter.
  