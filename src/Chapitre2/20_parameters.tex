% !TEX root=/home/tavant/these/manuscript/src/manuscript.tex

\section{Presentation of study}
  \label{sec-params}
  
  \begin{Chabstract}
    
  This study is based on the preliminary work of V. Croes, that he has presented in his Ph.D. thesis \citep{croes2017}.
  Based on this results, I continued the analyze which has been published in  \citet*{tavant2018}.
\end{Chabstract}
  
  \vspace{1ex}
  
  
  As introduced in \Cref{ch-1}, the \ac{HET} behavior is dependent of the axial electron transport toward the anode across the magnetic barrier.
  Two main phenomena are proposed to enhance the electron mobility,
  \begin{itemize}
    \item plasma instabilities and in particular the azimuthal \ac{ECDI}
    \item the electron induced electron emission from the wall
  \end{itemize}
  In order to compare quantitatively the relative importance of the two phenomena, we propose to conduct a parametric study on the dielectric wall characteristics.

  As highlighted before, the \ac{ECDI} rises due to the $E \times B$ electron drift, but saturates thanks to the axial convection model.
  Before investigating the time dependent behavior of the system, we focus in this \Cref{ch-2} on the average values at steady-state.
  The first section describes the parameters of the simulation, 
  while the second section highlights the main characteristics of the simulation results.

  
  The simulation domain corresponds to the exit plane of the thruster.
  Hence, a neutral pressure $P_n$ of 0.1~mTorr and a plasma density $n_e$ of $\sn{1}{17}$ m$^{-3}$ are used.
  The fixed axial electric field and radial magnetic field are $E_z=\sn{2}{4}\,\volt\per\meter$ s and $B_r=200$ G, respectively.
  The rectangular \ac{2D} domain measures $L_r=2$~cm in the radial dimension and $L_{\theta}=0.5$~cm in the azimuthal direction.
  The axial length used for the convection is fixed at $L_z=1$~cm.
  It is important to note that the results shown in this chapter have been obtained at the beginning of my thesis, before the study of the convection presented in \cref{ch-1}.
  Hence, in this chapter we use the convection model of \citet{lafleur2016a}.
  However, we have validated at posteriori that the convection model used does not modify the results under the conditions studied.
  
  The numerical parameters are chosen to respect the stability criterion of \ac{PIC} simulation, and are presented in \Cref{parameters}
  
  \begin{table}[htbp] %PIC parameters
       \centering
       \ra{1.3}
       \caption{\label{parameters} Standard operating and numerical parameters used in the 2D PIC simulations of an HET.  The simulation results are given as representative values.}
       \begin{tabular}{@{}r c c c@{}} 
          \toprule
          {\bf Physical Parameter} & notation & Value & Unit \\
          \midrule
          Gas & & Xenon & - \\
          Domain dimensions & $L_{x} \times L_{y} \times L_{z}$ & $2.0 \times 0.5 \times 1.0$ & [cm$^3$] \\
          Radial magnetic field & $B_{0}$                    & $200$                 & [{G}] \\
          Axial electric field & $E_{0}$                    & $2 \times 10^{4}$     & [{Vm}$^{-1}$] \\
          Mean plasma density & $n_{0}$                    & $3 \times 10^{17}$    & [{m}$^{-3}$] \\
          Initial electron temperature & $\Te_{,0}  $               & $5.0$                 & [{V}] \\
          Initial ion temperature & $T_{i,0}   $               & $0.1$                 & [{V}] \\
          Secondary electron temperature & $T_{see}   $               & $1.0$                 & [{V}] \\
          Neutral gas pressure & $P_{n}     $               & $1.0$                 & [{mTorr}] \\
          Neutral gas temperature & $T_{n}     $               & $300$                 & [{K}] \\
          Neutral gas density & $n_{g}     $               & $3.22 \times 10^{19}$ & [{m}$^{-3}$]\\
          \midrule
          {\bf Simulation Parameter} &  &   &  \\
          
          Time step & $\Delta t  $                      & $4 \times 10^{-12}$ & [{s}] \\
          Cell size & $\Delta x = \Delta y = \Delta z $ & $2 \times 10^{-5}$  & [{m}] \\
          Number of particles per cell & $N/NG      $                      & $80$                & [{part/cell}] \\
          \midrule
          {\bf Typical quantities} &  &  &  \\ 
          Electron plasma frequency & $\omega_{pe}$               & $3.1 \times 10^{10} $  & [rad/s]\\
          Iopn plasma frequency & $\omega_{pi}$               & $36 \times 10^{6} $  & [rad/s]\\
          Electron cyclotron frequency & $\omega_{ce}$               &  $3.5\times 10^{9}$  & [rad/s] \\
          Electron Larmor radius & $r_{Le}$                    & 6$\times 10^{-4}$    & [m] \\
          \bottomrule
       \end{tabular}
    \end{table}
  
  
  The simulation is initialized with a uniform density of particles, following a Maxwellian distribution for temperature $\Te_{,0}$ and $\Ti_{,0}$ for the electrons and the ions respectively.
  
  