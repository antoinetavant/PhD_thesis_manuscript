% !TEX root=/home/tavant/these/manuscript/src/manuscript.tex

\section{Impact of the radial boundary conditions on the oscillations}
  \label{subsec-BC}

  In \Cref{sec-DR-BC}, we discussed the choice of the radial wavenumber of the instability observed.
  Changing the radial electric boundary condition could affect the instability.
  Therefore, we discuss in this section  the impacts of the dielectric electrostatic boundary condition on the oscillations.
  We have seen in \Cref{sec-diel_layer} that the dielectric boundary does not affect the macroscopic simulation results.
  \Cref{fig-closswallosci} shows the radial evolution in the first few cells from the wall of the amplitude of the oscillation of the azimuthal electric field on the left, and the ion density on the right, with grounded (metallic) wall and dielectric wall.
  
  \begin{figure}[!hbt]
    \centering
    \includegraphics[width=0.8\textwidth]{Ex_closewall.pdf}
    \caption{Radial evolution in the first cells of the amplitude of the oscillation of (left) the azimuthal electric field and (right) the ion density, with grounded (metallic) wall and dielectric wall.}
    \label{fig-closswallosci}
  \end{figure}
  
  We can see in \cref{fig-closswallosci} that the boundary condition does not affect the ion oscillations.
  This is consistent with the observation made in \cref{subsec-kr} that the ion fluctuation was not affected by the wall.
  On the other hand, the azimuthal electric field has to go to zero when the wall is grounded, which is not the case with a dielectric layer.
  
  Nevertheless, the difference in $E_{\theta}$ between the two boundary conditions quickly disappears inside the plasma domain.
  Indeed, after a dozen cells from the wall, corresponding to a few Debye lengths, the amplitudes of $\delta E_{\theta}$ are equal for both cases.
  Hence, the electrostatic boundary condition induces only minor differences on the instability and therefore on the characteristics of  the plasma discharge.
  
  