% !TEX root=/home/tavant/these/manuscript/src/manuscript.tex




\chapter{Parametric study of the dielectric characteristics}
\label{ch-2}
Structure :

{\bf II. Parametric study of the dielectric} 30 pages
\begin{zzz}
  This chapter takes the 1rst paper which uses Vivien's results.

  2.1 Fully metallic wall (no SEE, grounded).

  2.2 Impact of Dielectric layer without SEE

  2.3 Impact of SEE with grounded wall

  2.4 SEE and dielectric in the same time

  2.5 Discrepancy between $\mean{\Te}$, $\sigma_{PIC}$ and $\sigma_{theo} = \sigma_0 + (1 - \sigma_0) \frac{2 T_e}{\epsilon_0}$
\end{zzz}


In \Cref{ch-1}, we describe the simulation domain and the numerical models used.
As highlighted, the \ac{ECDI} rises due to the $E \times B$ electron drift, but saturate thanks to the axial convection model.
Before investigating the time dependent characteristics of the system, we focus in this \Cref{ch-2} on the average values at steady state.
A parametric study has been conducted on the radial boundary condition, on both the dielectric layer model in the Poisson equation, and the electron emission.


% !TEX root=/home/tavant/these/manuscript/src/manuscript.tex

\section{Canonical simulation results}
  \label{sec-canonical}
  
  The {\it canonical} case is the reference case that will be extensively described and commented.
  It will be used then to analyse and quantify the effects of the two aspect of the dielectric walls:
  \begin{itemize}
    \item the dielectric layer on the plasma potential, analysed in  % \cref{sec-dielectric_layer}
    \item the electron emission, analysed in %\cref{sec-see}
  \end{itemize}
  
  
  \Cref{fig-profiles} shows the radial profiles of the electron and ions densities averaged azimuthally.
  We can see the sheath close to the wall where the electron density fall rapidly compare to the ions.
  
  
  \begin{figure}[hbtp]
    \centering
    \includegraphics[width=\defaultwidth]{density_profile.pdf}
    \caption{Radial profile of the ion and electron densities at steady state.}
    \label{fig-profiles}
  \end{figure}
% !TEX root=/home/tavant/these/manuscript/src/manuscript.tex

\section{Conclusion}
  \label{sec-conclusion_ch2}
  
